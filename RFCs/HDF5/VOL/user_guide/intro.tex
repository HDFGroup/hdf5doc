
%%% Local Variables: 
%%% mode: latex
%%% TeX-master: t
%%% End: 

\section{Introduction}
The Virtual Object Layer (VOL) is an abstraction layer in the HDF5
library that intercepts all API calls that could potentially access
objects in an HDF5 container and forwards those calls to plugin
''object drivers''. The plugins could store the objects in variety of
ways. A plugin could, for example, have objects be distributed
remotely over different platforms, provide a raw mapping of the model
to the file system, or even store the data in other file formats (like
native netCDF or HDF4 format). The user still gets the same data model
where access is done to a single HDF5 “container”; however the plugin
object driver translates from what the user sees to how the data is
actually stored. Having this abstraction layer maintains the object
model of HDF5 and would allow HDF5 developers or users to write their
own plugins for accessing HDF5 data.

This user guide is for developers interested in developing a VOL
plugin for the HDF5 library. The document is meant to be used in
conjunction with the HDF5 reference manual. It is assumed that the
reader has good knowledge of the VOL architecture obtained by reading
the VOL architectural design document~\cite{vol:rfc}. The
document will cover the steps needed to create external and internal
VOL plugins. Both ways have a lot of common steps and rules that will
be covered first.
