\section{New VOL API Routines}
\label{sec:api}
API routines have been added to the HDF5 library to manage VOL connectors. This section details each new API call and explains its intended usage. Additionally, a set of API calls that map directly to the VOL callbacks themselves have been added to aid in the development of passthrough connectors which can be stacked and/or split. A list of these API calls is given in an appendix.

\bigskip

\subsection{H5VLpublic.h}

The API calls in this header are for VOL management and general use (i.e., not
limited to VOL connector authors).

\subsubsection{H5VLregister\_connector\_by\_name}
\begin{mdframed}[style=bgbox]
\textbf{Signature:}
\begin{lstlisting}
hid_t H5VLregister_by_name(const char *connector_name, hid_t vipl_id);
\end{lstlisting}
\textbf{Arguments:}\\
\begin{tabular}{l p{13.5cm}}
  {\tt name} & (IN): The connector name to search for and register.\\
  {\tt vipl\_id} & (IN): An ID for a VOL initialization property list (vipl).\\
\end{tabular}
\end{mdframed}
Registers a VOL connector with the HDF5 library given the name of the connector and returns an identifier for it (\texttt{H5I\_INVALID\_HID} on errors). If the connector is already registered, the library will create a new identifier for it and returns it to the user; otherwise the library will search the plugin path for a connector of that name, loading and registering it, returning an ID for it, if found. See the VOL User Guide for more information on loading plugins and the search paths.
\bigskip


\subsubsection{H5VLregister\_connector\_by\_value}
\begin{mdframed}[style=bgbox]
\textbf{Signature:}
\begin{lstlisting}
hid_t H5VLregister_by_value(H5VL_class_value_t connector_value, hid_t vipl_id);
\end{lstlisting}
\textbf{Arguments:}\\
\begin{tabular}{l p{13.5cm}}
  {\tt connector\_value} & (IN): The connector value to search for and register.\\
  {\tt vipl\_id} & (IN): An ID for a VOL initialization property list (vipl).\\
\end{tabular}
\end{mdframed}
Registers a VOL connector with the HDF5 library given a value for the connector and returns an identifier for it (\texttt{H5I\_INVALID\_HID} on errors). If the connector is already registered, the library will create a new identifier for it and returns it to the user; otherwise the library will search the plugin path for a connector of that name, loading and registering it, returning an ID for it, if found. See the VOL User Guide for more information on loading plugins and the search paths.
\bigskip


\subsubsection{H5VLis\_connector\_registered\_by\_name}
\begin{mdframed}[style=bgbox]
\textbf{Signature:}
\begin{lstlisting}
htri_t H5VLis_connector_registered_by_name(const char *name);
\end{lstlisting}
\textbf{Arguments:}\\
\begin{tabular}{l p{13.5cm}}
  {\tt name} & (IN): The connector name to check for.\\
\end{tabular}
\end{mdframed}
Checks if a VOL connector is registered with the library given the connector name and returns TRUE/FALSE on success, otherwise it returns a negative value.
\bigskip


\subsubsection{H5VLis\_connector\_registered\_by\_value}
\begin{mdframed}[style=bgbox]
\textbf{Signature:}
\begin{lstlisting}
htri_t H5VLis_connector_registered_by_value(H5VL_class_value_t connector_value);
\end{lstlisting}
\textbf{Arguments:}\\
\begin{tabular}{l p{13.5cm}}
  {\tt value} & (IN): The connector value to check for.\\
\end{tabular}
\end{mdframed}
Checks if a VOL connector is registered with the library given the connector value and returns TRUE/FALSE on success, otherwise it returns a negative value.
\bigskip


\subsubsection{H5VLget\_connector\_id}
\begin{mdframed}[style=bgbox]
\textbf{Signature:}
\begin{lstlisting}
hid_t H5VLget_connector_id(hid_t obj_id);
\end{lstlisting}
\textbf{Arguments:}\\
\begin{tabular}{l p{13.5cm}}
  {\tt id} & (IN): An ID for an HDF5 VOL object.\\
\end{tabular}
\end{mdframed}
Given a VOL object such as a dataset or an attribute, this function returns an identifier for its associated connector. If the ID is not a VOL object (such as a dataspace or uncommitted datatype), \texttt{H5I\_INVALID\_HID} is returned. The identifier must be released with a call to {\tt H5VLclose()}.
\bigskip


\subsubsection{H5VLget\_connector\_id\_by\_name}
\begin{mdframed}[style=bgbox]
\textbf{Signature:}
\begin{lstlisting}
hid_t H5VLget_connector_id_by_name(const char *name);
\end{lstlisting}
\textbf{Arguments:}\\
\begin{tabular}{l p{13.5cm}}
  {\tt name} & (IN): The connector name to check for.\\
\end{tabular}
\end{mdframed}
Given a connector name that is registered with the library, this function returns an identifier for the connector. If the connector is not registered with the library, \texttt{H5I\_INVALID\_HID} is returned. The identifier must be released with a call to {\tt H5VLclose()}.
\bigskip


\subsubsection{H5VLget\_connector\_id\_by\_value}
\begin{mdframed}[style=bgbox]
\textbf{Signature:}
\begin{lstlisting}
hid_t H5VLget_connector_id_by_value(H5VL_class_value_t connector_value);
\end{lstlisting}
\textbf{Arguments:}\\
\begin{tabular}{l p{13.5cm}}
  {\tt value} & (IN): The connector value to check for.\\
\end{tabular}
\end{mdframed}
Given a connector value that is registered with the library, this function returns an identifier for the connector. If the connector is not registered with the library, \texttt{H5I\_INVALID\_HID} is returned. The identifier must be released with a call to {\tt H5VLclose()}.
\bigskip


\subsubsection{H5VLget\_connector\_name}
\begin{mdframed}[style=bgbox]
\textbf{Signature:}
\begin{lstlisting}
ssize_t H5VLget_connector_name(hid_t id, char *name/*out*/, size_t size);
\end{lstlisting}
\textbf{Arguments:}\\
\begin{tabular}{l p{13.5cm}}
  {\tt id} & (IN): The object identifier to check.\\
  {\tt name} & (OUT): Buffer pointer to put the connector name. If NULL, the library just returns the size required to store the connector name.\\
  {\tt size} & (IN): the size of the passed in buffer.\\
\end{tabular}
\end{mdframed}
Retrieves the name of a VOL connector given an object identifier that was created/opened with it. On success, the name length is returned.
\bigskip


\subsubsection{H5VLclose}
\begin{mdframed}[style=bgbox]
\textbf{Signature:}
\begin{lstlisting}
herr_t H5VLclose(hid_t connector_id);
\end{lstlisting}
\textbf{Arguments:}\\
\begin{tabular}{l p{13.5cm}}
 {\tt connector\_id} & (IN): A valid identifier of the connector to close.\\
\end{tabular}
\end{mdframed}
Shuts down access to the connector that the identifier points to and release resources associated with it.
\bigskip


\subsubsection{H5VLunregister\_connector}
\begin{mdframed}[style=bgbox]
\textbf{Signature:}
\begin{lstlisting}
herr_t H5VLunregister(hid_t connector_id);
\end{lstlisting}
\textbf{Arguments:}\\
\begin{tabular}{l p{13.5cm}}
  {\tt connector\_id} & (IN): A valid identifier of the connector to unregister.\\
\end{tabular}
\end{mdframed}
Unregisters a connector from the library and return a positive value on success otherwise return a negative value. The native VOL connector cannot be unregistered (this will return a negative herr\_t value).
\bigskip


\subsubsection{H5VLquery\_optional}
\begin{mdframed}[style=bgbox]
\textbf{Signature:}
\begin{lstlisting}
herr_t H5VLquery_optional(hid_t obj_id, H5VL_subclass_t subcls, int opt_type, uint64_t *flags);
\end{lstlisting}
\textbf{Arguments:}\\
\begin{tabular}{l p{13.5cm}}
  {\tt obj\_id} & (IN): A valid identifier of a VOL-managed object.\\
  {\tt subcls} & (IN): The subclass of the optional operation.\\
  {\tt opt\_type} & (IN): The optional operation. The native VOL connector uses hard-coded values. Other VOL connectors get this value when the optional operations are registered.\\
  {\tt flags} & (OUT): Bitwise flags indicating support and behavior.\\
\end{tabular}
\end{mdframed}
Determines if a connector or connector stack (determined from the passed-in object) supports an optional operation. The returned flags (listed below) not only indicate whether the operation is supported or not, but also give a sense of the option's behavior (useful for pass-through connectors).

Bitwise query flag values:
\begin{lstlisting}
#define H5VL_OPT_QUERY_SUPPORTED       0x0001 /* VOL connector supports this operation */
#define H5VL_OPT_QUERY_READ_DATA       0x0002 /* Operation reads data for object */
#define H5VL_OPT_QUERY_WRITE_DATA      0x0004 /* Operation writes data for object */
#define H5VL_OPT_QUERY_QUERY_METADATA  0x0008 /* Operation reads metadata for object */
#define H5VL_OPT_QUERY_MODIFY_METADATA 0x0010 /* Operation modifies metadata for object */
#define H5VL_OPT_QUERY_COLLECTIVE                                                                            \
    0x0020 /* Operation is collective (operations without this flag are assumed to be independent) */
#define H5VL_OPT_QUERY_NO_ASYNC  0x0040 /* Operation may NOT be executed asynchronously */
#define H5VL_OPT_QUERY_MULTI_OBJ 0x0080 /* Operation involves multiple objects */

\end{lstlisting}
\bigskip


\subsection{H5VLconnector.h}

This functionality is intended for VOL connector authors and includes helper
functions that are useful for writing connectors.

API calls to manage optional operations are also found in this header file.
These are discussed in the section on optional operations, above.

\subsubsection{H5VLregister\_connector}
\begin{mdframed}[style=bgbox]
\textbf{Signature:}
\begin{lstlisting}
hid_t H5VLregister_connector(const H5VL_class_t *cls, hid_t vipl_id);
\end{lstlisting}
\textbf{Arguments:}\\
\begin{tabular}{l p{13.5cm}}
  {\tt cls} & (IN): A pointer to the connector structure to register.\\
  {\tt vipl\_id} & (IN): An ID for a VOL initialization property list (vipl).\\
\end{tabular}
\end{mdframed}
Registers a user-defined VOL connector with the HDF5 library and returns an identifier for that connector (\texttt{H5I\_INVALID\_HID} on errors). This function is used when the application has direct access to the connector it wants to use and is able to obtain a pointer for the connector structure to pass to the HDF5 library.
\bigskip


\subsubsection{H5VLobject}
\begin{mdframed}[style=bgbox]
\textbf{Signature:}
\begin{lstlisting}
  void *H5VLobject(hid_t obj_id); 
\end{lstlisting}
\textbf{Arguments:}\\
\begin{tabular}{l p{13.5cm}}
 {\tt obj\_id} & (IN): identifier of the object to dereference.\\
\end{tabular}
\end{mdframed}
Retrieves a pointer to the VOL object from an HDF5 file or object identifier.
\bigskip


\subsubsection{H5VLget\_file\_type}
\begin{mdframed}[style=bgbox]
\textbf{Signature:}
\begin{lstlisting}
  hid_t H5VLget_file_type(void *file_obj, hid_t connector_id, hid_t dtype_id); 
\end{lstlisting}
\textbf{Arguments:}\\
\begin{tabular}{l p{13.5cm}}
 {\tt file\_obj} & (IN): pointer to a file or file object's connector-specific data.\\
 {\tt connector\_id} & (IN): A valid identifier of the connector to use.\\
 {\tt dtype\_id} & (IN): A valid identifier for the type.\\
\end{tabular}
\end{mdframed}
Returns a copy of the {\tt dtype\_id} parameter but with the location set to
be in the file. Returns a negative value ({\tt H5I\_INVALID\_HID}) on errors.
\bigskip


\subsubsection{H5VLpeek\_connector\_id\_by\_name}
\begin{mdframed}[style=bgbox]
\textbf{Signature:}
\begin{lstlisting}
  hid_t H5VLpeek_connector_id_by_name(const char *name); 
\end{lstlisting}
\textbf{Arguments:}\\
\begin{tabular}{l p{13.5cm}}
 {\tt name} & (IN): name of the connector to query.\\
\end{tabular}
\end{mdframed}
Retrieves the ID for a registered VOL connector based on a connector name.
This is done without duplicating
the ID and transferring ownership to the caller (as it normally the case in the
HDF5 library). The ID returned from this operation should not be closed.
This is intended for use by VOL connectors to find their own ID.
Returns a negative value ({\tt H5I\_INVALID\_HID}) on errors.
\bigskip


\subsubsection{H5VLpeek\_connector\_id\_by\_value}
\begin{mdframed}[style=bgbox]
\textbf{Signature:}
\begin{lstlisting}
  hid_t H5VLpeek_connector_id_by_value(H5VL_class_value_t value); 
\end{lstlisting}
\textbf{Arguments:}\\
\begin{tabular}{l p{13.5cm}}
 {\tt value} & (IN): value of the connector to query.\\
\end{tabular}
\end{mdframed}
Retrieves the ID for a registered VOL connector based on a connector value.
This is done without duplicating
the ID and transferring ownership to the caller (as it normally the case in the
HDF5 library). The ID returned from this operation should not be closed.
This is intended for use by VOL connectors to find their own ID.
Returns a negative value ({\tt H5I\_INVALID\_HID}) on errors.
\bigskip


\subsection{H5VLconnector\_passthru.h}

This functionality is intended for VOL connector authors who are writing
pass-through connectors and includes helper functions that are useful for
writing such connectors. Callback equivalent functions can be found in this
header as well. A list of these functions is included as an appendix to this
document.


\subsubsection{H5VLcmp\_connector\_cls}
\begin{mdframed}[style=bgbox]
\textbf{Signature:}
\begin{lstlisting}
  herr_t H5VLcmp_connector_cls(int *cmp, hid_t connector_id1, hid_t connector_id2);
\end{lstlisting}
\textbf{Arguments:}\\
\begin{tabular}{l p{13.5cm}}
 {\tt cmp} & (OUT): a value like strcmp.\\
 {\tt connector\_id1} & (IN): the ID of the first connector to compare.\\
 {\tt connector\_id2} & (IN): the ID of the second connector to compare.\\
\end{tabular}
\end{mdframed}
Compares two connectors (given by the connector IDs) to see if they refer to the
same connector underneath.
Returns a positive value on success and a negative value on errors.
\bigskip


\subsubsection{H5VLwrap\_register}
\begin{mdframed}[style=bgbox]
\textbf{Signature:}
\begin{lstlisting}
  hid_t H5VLwrap_register(void *obj, H5I_type_t type);
\end{lstlisting}
\textbf{Arguments:}\\
\begin{tabular}{l p{13.5cm}}
 {\tt cmp} & (IN): an object to wrap.\\
 {\tt type} & (IN): the type of the object (see below).\\
\end{tabular}
\end{mdframed}
Wrap an internal object with a "wrap context" and register and return an
hid\_t for the resulting object. This routine is mainly targeted toward wrapping
objects for iteration routine callbacks (i.e. the callbacks from H5Aiterate*,
H5Literate* / H5Lvisit*, and H5Ovisit*). The type must be a VOL-managed object
class (H5I\_FILE, H5I\_GROUP, H5I\_DATATYPE, H5I\_DATASET, H5I\_MAP, or H5I\_ATTR).
Returns a negative value ({\tt H5I\_INVALID\_HID}) on errors.
\bigskip


\subsubsection{H5VLretrieve\_lib\_state}
\begin{mdframed}[style=bgbox]
\textbf{Signature:}
\begin{lstlisting}
  herr_t H5VLretrieve_lib_state(void **state);
\end{lstlisting}
\textbf{Arguments:}\\
\begin{tabular}{l p{13.5cm}}
 {\tt state} & (OUT): the library state.\\
\end{tabular}
\end{mdframed}
Retrieves a copy of the internal state of the HDF5 library,
so that it can be restored later.
Returns a positive value on success and a negative value on errors.
\bigskip


\subsubsection{H5VLstart\_lib\_state}
\begin{mdframed}[style=bgbox]
\textbf{Signature:}
\begin{lstlisting}
  herr_t H5VLstart_lib_state(void);
\end{lstlisting}
\end{mdframed}
Opens a new internal state for the HDF5 library.
Returns a positive value on success and a negative value on errors.
\bigskip


\subsubsection{H5VLrestore\_lib\_state}
\begin{mdframed}[style=bgbox]
\textbf{Signature:}
\begin{lstlisting}
  herr_t H5VLrestore_lib_state(const void *state);
\end{lstlisting}
\textbf{Arguments:}\\
\begin{tabular}{l p{13.5cm}}
 {\tt state} & (IN): the library state.\\
\end{tabular}
\end{mdframed}
Restores the internal state of the HDF5 library.
Returns a positive value on success and a negative value on errors.
\bigskip


\subsubsection{H5VLfinish\_lib\_state}
\begin{mdframed}[style=bgbox]
\textbf{Signature:}
\begin{lstlisting}
  herr_t H5VLfinish_lib_state(void);
\end{lstlisting}
\end{mdframed}
Closes the state of the library, undoing the effects of {\tt H5VLstart\_lib\_state}.
Returns a positive value on success and a negative value on errors.
\bigskip


\subsubsection{H5VLfree\_lib\_state}
\begin{mdframed}[style=bgbox]
\textbf{Signature:}
\begin{lstlisting}
  herr_t H5VLfree_lib_state(const void *state);
\end{lstlisting}
\textbf{Arguments:}\\
\begin{tabular}{l p{13.5cm}}
 {\tt state} & (IN): the library state.\\
\end{tabular}
\end{mdframed}
Free a retrieved library state.
Returns a positive value on success and a negative value on errors.
\bigskip


%%% Local Variables: 
%%% mode: latex
%%% TeX-master: t
%%% End: 
