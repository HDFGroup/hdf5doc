\section{New VOL API Routines}
\label{sec:api}
New API routines have been added to the HDF5 library to manage VOL connectors. This section details each new API call and explains their intended usage. Additionally, a set of API calls that map directly to the VOL callbacks themselves have been added to aid in the development of passthrough connectors which can be stacked and/or split. A list of these API calls is given in an appendix.

\bigskip

\subsection{H5VLpublic.h}

The API calls in this header are for VOL management and general use (i.e., not
limited to VOL connector authors).

\subsubsection{H5VLregister\_connector}
\begin{mdframed}[style=bgbox]
\textbf{Signature:}
\begin{lstlisting}
hid_t H5VLregister_connector(const H5VL_class_t *cls, hid_t vipl_id);
\end{lstlisting}

\textbf{Arguments:}\\
\begin{tabular}{l p{13.5cm}}
  {\tt cls} & (IN): A pointer to the connector structure to register.\\
  {\tt vipl\_id} & (IN): An ID for a VOL initialization property list (vipl).\\
\end{tabular}
\end{mdframed}

Registers a user-defined VOL connector with the HDF5 library and returns an identifier for that connector (\texttt{H5I\_INVALID\_HID} on errors). This function is used when the application has direct access to the connector it wants to use and is able to obtain a pointer for the connector structure to pass to the HDF5 library.
\bigskip

\subsubsection{H5VLregister\_connector\_by\_name}
\begin{mdframed}[style=bgbox]
\textbf{Signature:}
\begin{lstlisting}
hid_t H5VLregister_by_name(const char *connector_name, hid_t vipl_id);
\end{lstlisting}
\textbf{Arguments:}\\
\begin{tabular}{l p{13.5cm}}
  {\tt name} & (IN): The connector name to search for and register.\\
  {\tt vipl\_id} & (IN): An ID for a VOL initialization property list (vipl).\\
\end{tabular}
\end{mdframed}
Registers a VOL connector with the HDF5 library given the name of the connector and returns an identifier for it (\texttt{H5I\_INVALID\_HID} on errors). If the connector is already registered, the library will create a new identifier for it and returns it to the user; otherwise the library will search the plugin path for a connector of that name, loading and registering it, returning an ID for it, if found. See the VOL User Guide for more information on loading plugins and the search paths.
\bigskip

\subsubsection{H5VLregister\_connector\_by\_value}
\begin{mdframed}[style=bgbox]
\textbf{Signature:}
\begin{lstlisting}
hid_t H5VLregister_by_value(H5VL_class_value_t connector_value, hid_t vipl_id);
\end{lstlisting}
\textbf{Arguments:}\\
\begin{tabular}{l p{13.5cm}}
  {\tt connector\_value} & (IN): The connector value to search for and register.\\
  {\tt vipl\_id} & (IN): An ID for a VOL initialization property list (vipl).\\
\end{tabular}
\end{mdframed}
Registers a VOL connector with the HDF5 library given a value for the connector and returns an identifier for it (\texttt{H5I\_INVALID\_HID} on errors). If the connector is already registered, the library will create a new identifier for it and returns it to the user; otherwise the library will search the plugin path for a connector of that name, loading and registering it, returning an ID for it, if found. See the VOL User Guide for more information on loading plugins and the search paths.
\bigskip


\subsubsection{H5VLis\_connector\_registered}
\begin{mdframed}[style=bgbox]
\textbf{Signature:}
\begin{lstlisting}
htri_t H5VLis_connector_registered(const char *name);
\end{lstlisting}
\textbf{Arguments:}\\
\begin{tabular}{l p{13.5cm}}
  {\tt name} & (IN): The connector name to check for.\\
\end{tabular}
\end{mdframed}
Checks if a VOL connector is registered with the library given the connector name and returns TRUE/FALSE on success, otherwise it returns a negative value.\bigskip


\subsubsection{H5VLget\_connector\_id}
\begin{mdframed}[style=bgbox]
\textbf{Signature:}
\begin{lstlisting}
hid_t H5VLget_connector_id(const char *name);
\end{lstlisting}
\textbf{Arguments:}\\
\begin{tabular}{l p{13.5cm}}
  {\tt name} & (IN): The connector name to check for.\\
\end{tabular}
\end{mdframed}
Given a connector name that is registered with the library, this function returns an identifier for the connector. If the connector is not registered with the library, a negative value is returned. The identifier must be released with a call to {\tt H5VLclose()}.\bigskip


\subsubsection{H5VLget\_connector\_name}
\begin{mdframed}[style=bgbox]
\textbf{Signature:}
\begin{lstlisting}
ssize_t H5VLget_connector_name(hid_t id, char *name/*out*/, size_t size);
\end{lstlisting}

\textbf{Arguments:}\\
\begin{tabular}{l p{13.5cm}}
  {\tt id} & (IN): The object identifier to check.\\
  {\tt name} & (OUT): Buffer pointer to put the connector name. If NULL, the library just returns the size required to store the connector name.\\
  {\tt size} & (IN): the size of the passed in buffer.\\
\end{tabular}
\end{mdframed}
Retrieves the name of a VOL connector given an object identifier that was created/opened with it. On success, the name length is returned.\bigskip


\subsubsection{H5VLclose}
\begin{mdframed}[style=bgbox]
\textbf{Signature:}
\begin{lstlisting}
herr_t H5VLclose(hid_t connector_id);
\end{lstlisting}

\textbf{Arguments:}\\
\begin{tabular}{l p{13.5cm}}
 {\tt connector\_id} & (IN): A valid identifier of the connector to close.\\
\end{tabular}
\end{mdframed}
Shuts down access to the connector that the identifier points to and release resources associated with it.\bigskip


\subsubsection{H5VLunregister\_connector}
\begin{mdframed}[style=bgbox]
\textbf{Signature:}
\begin{lstlisting}
herr_t H5VLunregister(hid_t connector_id);
\end{lstlisting}
\textbf{Arguments:}\\
\begin{tabular}{l p{13.5cm}}
  {\tt connector\_id} & (IN): A valid identifier of the connector to unregister.\\
\end{tabular}
\end{mdframed}
Unregisters a connector from the library and return a positive value on success otherwise return a negative value. The native VOL connector cannot be unregistered (this will return a negative herr\_t value).\bigskip


\subsection{H5VLconnector.h}

This functionality is intended for VOL connector authors and includes helper
functions that are useful for writing connectors. Callback equivalent
functions can be found in {\tt H5VLconnector\_passthru.h}. A list of these
functions is included as an appendix to this document.

\subsubsection{H5VLobject}
\begin{mdframed}[style=bgbox]
\textbf{Signature:}
\begin{lstlisting}
  void *H5VLget_object(hid_t obj_id); 
\end{lstlisting}

\textbf{Arguments:}\\
\begin{tabular}{l p{13.5cm}}
 {\tt obj\_id} & (IN): identifier of the object to dereference.\\
\end{tabular}
\end{mdframed}
Retrieves a pointer to the VOL object from an HDF5 file or object identifier.\bigskip

\subsubsection{H5VLget\_file\_type}
\begin{mdframed}[style=bgbox]
\textbf{Signature:}
\begin{lstlisting}
  hid_t H5VLget_file_type(void *file_obj, hid_t connector_id, hid_t dtype_id); 
\end{lstlisting}

\textbf{Arguments:}\\
\begin{tabular}{l p{13.5cm}}
 {\tt file\_obj} & (IN): pointer to a file or file object's connector-specific data.\\
 {\tt connector\_id} & (IN): A valid identifier of the connector to use.\\
 {\tt dtype\_id} & (IN): A valid identifier for the type.\\
\end{tabular}
\end{mdframed}
Returns a copy of the {\tt dtype\_id} parameter but with the location set to
be in the file. Returns a negative value ({\tt H5I\_INVALID\_HID}) on errors.\bigskip

\subsubsection{H5VLpeek\_connector}
\begin{mdframed}[style=bgbox]
\textbf{Signature:}
\begin{lstlisting}
  hid_t H5VLpeek_connector_id(const char *name); 
\end{lstlisting}

\textbf{Arguments:}\\
\begin{tabular}{l p{13.5cm}}
 {\tt name} & (IN): name of the connector to query.\\
\end{tabular}
\end{mdframed}
Retrieves the ID for a registered VOL connector. This is done without duplicating
the ID and transferring ownership to the caller (as it normally the case in the
HDF5 library). The ID returned from this operation should not be closed.
This is intended for use by VOL connectors to find their own ID.

Returns a negative value ({\tt H5I\_INVALID\_HID}) on errors.\bigskip


%%% Local Variables: 
%%% mode: latex
%%% TeX-master: t
%%% End: 
