\section{Creating a New Connector}

\subsection{Overview}
Creating a new VOL connector can be a complicated process. You will need to map
your storage system to the HDF5 data model through the lens of the VOL and this
may involve some impedence mismatch that you will have to work around. The good
news is that the HDF5 library has been re-engineered to handle arbitrary,
connector-specific data structures via the VOL callbacks, so no knowledge of
the library internals is necessary to write a VOL connector.

Writing a VOL connector requires these things:

\begin{enumerate}
    \item Decide on library vs plugin vs internal.
    \item Set up your build/test files (CMake, Autotools, etc.).
    \item Fill in some boilerplate information in your {\tt H5VL\_class\_t} struct.
    \item Decide how you will perform any necessary initialization needed
        by your storage system.
    \item Map Storage to HDF5 File Objects
    \item Create implementations for the callbacks you need to support.
    \item Test the connector.
\end{enumerate}

Each of the steps listed above is decribed in more detail in this section of the document.

The "model then implement" steps can be performed iteratively. You might begin
by only supporting files, datasets, and groups and only allowing basic operations
on them. In some cases, this may be all that is needed. As your needs grow, you
can repeat those steps and increase the connector's HDF5 API coverage at a pace
that makes sense for your users.

Also, note that this document only covers writing VOL connectors using the
C programming language. It is often possible to write connectors in other programming languages
(e.g.; Python) via the language's C interop facilities, but that topic is out of
scope for this document.

\subsection{Sources Of Information}

Use of the VOL, including topics such as registration and loading VOL plugins, is
described in the VOL User Guide.

Public header files you will need to be familiar with include:

\begin{tabularx}{\linewidth}{ l X }
  \texttt{H5VLpublic.h} & Public VOL header.\\
  \texttt{H5VLconnector.h} & Main connector author header. Contains definitions for the main VOL struct and callbacks, enum values, etc.\\
  \texttt{H5VLconnector\_passthru.h} & Helper routines for passthrough connector authors.\\
    \texttt{H5VLnative.h} & Native VOL connector header. May be useful if your connector will attempt to implement native HDF5 API calls that are handled via the \textit{optional} callbacks.\\
  \texttt{H5PLextern.h} & Needed if your connector will be built as a plugin.\\
\end{tabularx}

Many VOL connectors can be found on The HDF Group's Bitbucket server, located at
\url{https://bitbucket.hdfgroup.org/projects/HDF5VOL}. Not all of these VOL
connectors are supported by THe HDF Group and the level of completeness varies,
but the connectors found there can serve as examples of working implementations.


\subsection{Library vs Plugin vs Internal}

When building a VOL connector, you have several options:

\subsubsection*{Library}

The connector can be built as a normal shared or static library. Software that
uses your connector will have to link to it just like any other library. This
can be convenient since you don't have to deal with plugin paths and searching
for the connector at runtime, but it also means that software which uses
your connector will have to be built and linked against it.

\subsubsection*{Plugin}

You can also build your connector as a dynamically loaded plugin. The mechanism
for this is the same mechanism used to dynamically load HDF5 filter plugins. This can
allow use of your connector via the VOL environment variable, without modifying
the application, but requires your plugin to be discoverable at runtime. See
the VOL User Guide for more information about using HDF5 plugins.

To build your connector as a plugin, you will have to include {\tt H5PLextern.h}
(a public header distributed with the library) and implement the
{\tt H5PLget\_plugin\_type()} and {\tt H5PLget\_plugin\_info()}
calls, both of which are trivial to code up. It also often requires your
connector to be built with certain compile/link options. The VOL connector
template does all of these things.

To show how easy this is to accomplish, here is the complete implementation
of those functions in the template VOL connector:

\begin{lstlisting}
H5PL_type_t H5PLget_plugin_type(void) {return H5PL_TYPE_VOL;}
const void *H5PLget_plugin_info(void) {return &template_class_g;}
\end{lstlisting}

The HDF5 library's plugin loading code will call {\tt H5PLget\_plugin\_type()}
to determine the type of plugin (e.g.; filter, VOL) and {\tt H5PLget\_plugin\_info()}
to get the class struct, which allows the library to query the plugin for its
name and value to see if it has found a requested plugin. When a match is found,
the library will use the class struct to register the connector and map its callbacks.

For the HDF5 library to be able to load an external plugin dynamically, the plugin developer has to define two public routines with the following name and signature:
\begin{lstlisting}
H5PL_type_t H5PLget_plugin_type(void)
const void *H5PLget_plugin_info(void)
\end{lstlisting}
{\tt H5PLget\_plugin\_type} should return the library type which should always be {\tt H5PL\_TYPE\_VOL}. {\tt H5PLget\_plugin\_info} should return a pointer to the plugin structure defining the VOL plugin with all the callbacks. For example, consider an external plugin defined as:
\begin{lstlisting}
static const H5VL_class_t H5VL_log_g = {
    1,	/* version */
    502,	/* value */
    "log",	/* name */
    ...
}
\end{lstlisting}
The plugin would implement the two routines as:
\begin{lstlisting}
H5PL_type_t H5PLget_plugin_type(void) {return H5PL_TYPE_VOL;}
const void *H5PLget_plugin_info(void) {return &H5VL_log_g;}
\end{lstlisting}

\subsubsection*{Internal}

Your VOL connector can also be constructed as a part of the HDF5 library. This
works in the same way as the stdio and multi virtual file drivers (VFDs) and
does not require knowledge of HDF5 internals or use of non-public API calls. You
simply have to add your connector's files to the {\tt Makefile.am} and/or
{\tt CMakeLists.txt} files in the source distribution's {\tt src} directory.
This requires maintaining a private build of the library, though, and is not
recommended.

\subsection{Build Files / VOL Template}

We have created a template VOL connector that includes both Autotools and CMake
build files. The constructed VOL connector includes no real functionality, but 
can be registered and loaded as a plugin.

The VOL template  can be found here:

\quad \quad \url{https://bitbucket.hdfgroup.org/projects/HDF5VOL/repos/template}

The purpose of this template is to quickly get you to the point where you can
begin filling in the callback functions and writing tests. You can copy this
code to your own repository to serve as the basis for your new connector.


\subsection{{\tt H5VL\_class\_t} Boilerplate}

Several fields in the {\tt H5VL\_class\_t} struct will need to be filled in.

In HDF5 1.12.0, {\tt version} will be 1, indicating version 1 of the {\tt H5VL\_class\_t} struct.

Every connector needs a {\tt name} and {\tt value}. The library will use these when
loading and registering the connector (as described in the VOL User Guide), so
they should be unique in your ecosystem.

VOL connector values are integers, with a maximum value of 65535. Values from
0 to 255 are reserved for internal use by The HDF Group. The native VOL connector
has a value of 0.

As is the case with HDF5 filters, The HDF Group can register the value you've
chosen for your VOL connector so that others don't use it. Please contact
\href{mailto:help@hdfgroup.org}{help@hdfgroup.org} for help with this. We currently
do not register connector names.

The {\tt cap\_flags} field is currently not fully utilized. In the future, this
will be most likely be used to indicate which callbacks and capabilities are
supported by the VOL connector, but for now this struct field contains few flags, 
a list of which can be found in {\tt H5VLconnector.h}.

\subsection{Initialization and Shutdown}

You'll need to decide how to perform any initialization and shutdown tasks that
are required by your connector. There are initialize and terminate callbacks
in the {\tt H5VL\_class\_t} struct to handle this. They are invoked when
the connector is registered and unregistered, respectively. If this is
unsuitable, you may have to create custom connector-specific API calls to handle
initialization and termination. It may also be useful to perform operations in
a custom API call used to set the VOL connector in the fapl.

\subsection{Map Storage to HDF5 File Objects}

The most difficult part of designing a new VOL connector is going to determining
how to support HDF5 file objects and operaions using your storage system.
There isn't much specific advice to give here, as
each connector will have unique needs, but a forthcoming "tutorial" connector
will set up a simple connector and demonstrate this process.


\subsection{Fill In VOL Callbacks}

For each file object you support in your connector (including the file itself),
you will need to create a data struct to hold whatever file object
metadata that are needed by your connector. For example, a data structure for
a VOL connector based on text files might have a file struct that contains a
file pointer for the text file, buffers used for cacheing data, etc. Pointers
to these data structures are where your connector's state is stored and are
returned to the HDF5 library from the create/open/etc. callbacks such as
\textit{dataset create}.

Once you have your data structures, you'll need to create your own
implementations of the callback functions and map them via your
{\tt H5VL\_class\_t} struct. 
\subsection{Testing Your Connector}

At the time of writing, some of the HDF5 library tests have been abstracted
out of the library with their native-file-format-only sections removed and
added to a VOL test suite available here:

\quad \quad \url{https://bitbucket.hdfgroup.org/projects/HDF5VOL/repos/vol-tests}

This is an evolving set of tests, so see the documentation in that repository
for instructions as to its use. You may want to clone and modify and/or
extend these tests for use with your own connector.

In the future, we plan to modify the HDF5 test suite that ships with the
library to use a future VOL capabilities flags scheme to selectively run
tests that a particular connector supports. As this is a large task, it may be
some time before that work is complete.

\subsection{Passthrough Connectors}

Coming Soon


%%% Local Variables: 
%%% mode: latex
%%% TeX-master: t
%%% End: 
