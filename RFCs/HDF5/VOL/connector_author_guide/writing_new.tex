\section{Creating a New Connector}

\subsection{Overview}
Creating a new VOL connector can be a complicated process. You will need to map
your storage system to the HDF5 data model through the lens of the VOL and this
may involve some impedence mismatch that you will have to work around. The good
news is that the HDF5 library has been re-engineered to handle arbitrary,
connector-specific data structures via the VOL callbacks, so no knowledge of
the library internals is necessary to write a VOL connector.

Writing a VOL connector requires these things:

\begin{enumerate}
    \item Decide on library vs plugin vs internal.
    \item Set up your build/test files (CMake, Autotools, etc.).
    \item Fill in some boilerplate information in your {\tt H5VL\_class\_t} struct.
    \item Decide how you will perform any necessary initialization needed
        by your storage system.
    \item Map Storage to HDF5 File Objects
    \item Create implementations for the callbacks you need to support.
    \item Test the connector.
\end{enumerate}

Each of the steps listed above is decribed in more detail in this section of the document.

The "model then implement" steps can be performed iteratively. You might begin
by only supporting files, datasets, and groups and only allowing basic operations
on them. In some cases, this may be all that is needed. As your needs grow, you
can repeat those steps and increase the connector's HDF5 API coverage at a pace
that makes sense for your users.

Also, note that this document only covers writing VOL connectors using the
C programming language. It is often possible to write connectors in other programming languages
(e.g.; Python) via the language's C interop facilities, but that topic is out of
scope for this document.

\subsection{The HDF5 1.12.x VOL Interface Is DEPRECATED}

Important changes were made to the VOL interface for HDF5 1.13.0 and, due to
binary compatibility issues, these cannot be merged to HDF5 1.12.x. For this
reason, VOL connector development should be shifted to target 1.13.0 as no
further development of the VOL interface will take place on the 1.12.x branch.
Unlike the other development branches of the library, there is no
{\tt hdf5\_1\_13} branch - all HDF5 1.13.0 development is taking place in
the {\tt develop} branch of the HDF5 repository and 1.13.x branches will split
off from that.

Note also that HDF5 1.13.0 is considered an unstable branch, and the API and
file format are subject to change ("unstable" means "not yet finalized", not "buggy").
The VOL feature is under active development and, although it is nearing its final form,
may change further before the stable HDF5 1.14.0 release targeted for 2022.

\subsection{VOL-Related HDF5 Header Files}

Use of the VOL, including topics such as registration and loading VOL plugins, is
described in the VOL User Guide.

Public header files you will need to be familiar with include:

\begin{tabularx}{\linewidth}{ l X }
  \texttt{H5VLpublic.h} & Public VOL header.\\
  \texttt{H5VLconnector.h} & Main header for connector authors. Contains definitions for the main VOL struct and callbacks, enum values, etc.\\
  \texttt{H5VLconnector\_passthru.h} & Helper routines for passthrough connector authors.\\
    \texttt{H5VLnative.h} & Native VOL connector header. May be useful if your connector will attempt to implement native HDF5 API calls that are handled via the \textit{optional} callbacks.\\
  \texttt{H5PLextern.h} & Needed if your connector will be built as a plugin.\\
\end{tabularx}

Many VOL connectors are listed on The HDF Group's VOL plugin registration page,
located at \url{https://portal.hdfgroup.org/display/support/Registered+VOL+Connectors}.
Not all of these VOL connectors are supported by The HDF Group and the level
of completeness varies, but the connectors found there can serve as examples
of working implementations.

\subsection{Library vs Plugin vs Internal}

When building a VOL connector, you have several options:

\subsubsection*{Library}

The connector can be built as a normal shared or static library. Software that
uses your connector will have to link to it just like any other library. This
can be convenient since you don't have to deal with plugin paths and searching
for the connector at runtime, but it also means that software which uses
your connector will have to be built and linked against it.

\subsubsection*{Plugin}

You can also build your connector as a dynamically loaded plugin. The mechanism
for this is the same mechanism used to dynamically load HDF5 filter plugins. This can
allow use of your connector via the VOL environment variable, without modifying
the application, but requires your plugin to be discoverable at runtime. See
the VOL User Guide for more information about using HDF5 plugins.

To build your connector as a plugin, you will have to include {\tt H5PLextern.h}
(a public header distributed with the library) and implement the
{\tt H5PLget\_plugin\_type()} and {\tt H5PLget\_plugin\_info()}
calls, both of which are trivial to code up. It also often requires your
connector to be built with certain compile/link options. The VOL connector
template does all of these things.

The HDF5 library's plugin loading code will call {\tt H5PLget\_plugin\_type()}
to determine the type of plugin (e.g.; filter, VOL) and {\tt H5PLget\_plugin\_info()}
to get the class struct, which allows the library to query the plugin for its
name and value to see if it has found a requested plugin. When a match is found,
the library will use the class struct to register the connector and map its callbacks.

For the HDF5 library to be able to load an external plugin dynamically, the plugin developer has to define two public routines with the following name and signature:
\begin{lstlisting}
H5PL_type_t H5PLget_plugin_type(void)
const void *H5PLget_plugin_info(void)
\end{lstlisting}

To show how easy this is to accomplish, here is the complete implementation
of those functions in the template VOL connector:

\begin{lstlisting}
H5PL_type_t H5PLget_plugin_type(void) {return H5PL_TYPE_VOL;}
const void *H5PLget_plugin_info(void) {return &template_class_g;}
\end{lstlisting}

{\tt H5PLget\_plugin\_type} should return the library type which should always be {\tt H5PL\_TYPE\_VOL}. {\tt H5PLget\_plugin\_info} should return a pointer to the plugin structure defining the VOL plugin with all the callbacks. For example, consider an external plugin defined as:

\begin{lstlisting}
static const H5VL_class_t H5VL_foo_g = {
    2,      /* version */
    12345,  /* value */
    "foo",  /* name */
    ...
}
\end{lstlisting}

The plugin would implement the two routines as:
\begin{lstlisting}
H5PL_type_t H5PLget_plugin_type(void) {return H5PL_TYPE_VOL;}
const void *H5PLget_plugin_info(void) {return &H5VL_foo_g;}
\end{lstlisting}

\subsubsection*{Internal}

Your VOL connector can also be constructed as a part of the HDF5 library. This
works in the same way as the stdio and multi virtual file drivers (VFDs) and
does not require knowledge of HDF5 internals or use of non-public API calls. You
simply have to add your connector's files to the {\tt Makefile.am} and/or
{\tt CMakeLists.txt} files in the source distribution's {\tt src} directory.
This requires maintaining a private build of the library, though, and is not
recommended.

\subsection{Build Files / VOL Template}

We have created a template terminal VOL connector that includes both Autotools and CMake
build files. The constructed VOL connector includes no real functionality, but 
can be registered and loaded as a plugin.

The VOL template  can be found here:

\quad \quad \url{https://github.com/HDFGroup/vol-template}

The purpose of this template is to quickly get you to the point where you can
begin filling in the callback functions and writing tests. You can copy this
code to your own repository to serve as the basis for your new connector.

A template passthrough VOL is also available. This will be discussed in the
section on passthrough connectors.

\subsection{{\tt H5VL\_class\_t} Boilerplate}

Several fields in the {\tt H5VL\_class\_t} struct will need to be filled in.

In HDF5 1.13.0, the {\tt version} field will be 2, indicating the connector
targets version 2 of the {\tt H5VL\_class\_t} struct. Version 1 of the struct
was never formally released and only available in the {\tt develop} branch of
the HDF5 git repository. Version 0 is used in the deprecated HDF5 1.12.x branch.

Every connector needs a {\tt name} and {\tt value}. The library will use these when
loading and registering the connector (as described in the VOL User Guide), so
they should be unique in your ecosystem.

VOL connector values are integers, with a maximum value of 65535. Values from
0 to 255 are reserved for internal use by The HDF Group. The native VOL connector
has a value of 0. Values of 256 to 511 are for connector testing and should
not be found in the wild. Values of 512 to 65535 are for external connectors.

As is the case with HDF5 filters, The HDF Group can assign you an official
VOL connector value. Please contact
\href{mailto:help@hdfgroup.org}{help@hdfgroup.org} for help with this. We currently
do not register connector names, though the name you've chosen will appear on
the registered VOL connectors page.

As noted above, registered VOL connectors will be listed at:

\url{https://portal.hdfgroup.org/display/support/Registered+VOL+Connectors}

A new {\tt conn\_version} field has been added to the class struct for 1.13.
This field is currently not used by the library so its use is determined by
the connector author. Best practices for this field will be determined in
the near future and this part of the guide will be updated.

The {\tt cap\_flags} field is used to determine the capabilities of the VOL
connector. At this time, the use of this field is limited to indicating
thread-safety, asynchronous capabilities, and ability to produce native
HDF5 files. Supported flags can be found in {\tt H5VLconnector.h}.

\begin{lstlisting}
/* Capability flags for connector */
#define H5VL_CAP_FLAG_NONE         0    /* No special connector capabilities */
#define H5VL_CAP_FLAG_THREADSAFE   0x01 /* Connector is threadsafe */
#define H5VL_CAP_FLAG_ASYNC        0x02 /* Connector performs operations asynchronously*/
#define H5VL_CAP_FLAG_NATIVE_FILES 0x04 /* Connector produces native file format */
\end{lstlisting}

\subsection{Initialization and Shutdown}

You'll need to decide how to perform any initialization and shutdown tasks that
are required by your connector. There are initialize and terminate callbacks
in the {\tt H5VL\_class\_t} struct to handle this. They are invoked when
the connector is registered and unregistered, respectively. The inialize
callback can take a VOL initialization property list, so any properties you
need for initialization can be applied to it. The HDF5 library currently makes
no use of the vipl so there are no default vipl properties.

If this is unsuitable, you may have to create custom connector-specific API calls to handle
initialization and termination. It may also be useful to perform operations in
a custom API call used to set the VOL connector in the fapl.

The initialization and terminate callbacks:
\begin{lstlisting}
herr_t (*initialize)(hid_t vipl_id); /**< Connector initialization callback */

herr_t (*terminate)(void);           /**< Connector termination callback */
\end{lstlisting}

\subsection{Map Storage to HDF5 File Objects}

The most difficult part of designing a new VOL connector is going to determining
how to support HDF5 file objects and operaions using your storage system.
There isn't much specific advice to give here, as
each connector will have unique needs, but a forthcoming "tutorial" connector
will set up a simple connector and demonstrate this process.


\subsection{Fill In VOL Callbacks}

For each file object you support in your connector (including the file itself),
you will need to create a data struct to hold whatever file object
metadata that are needed by your connector. For example, a data structure for
a VOL connector based on text files might have a file struct that contains a
file pointer for the text file, buffers used for cacheing data, etc. Pointers
to these data structures are where your connector's state is stored and are
returned to the HDF5 library from the create/open/etc. callbacks such as
\textit{dataset create}.

Once you have your data structures, you'll need to create your own
implementations of the callback functions and map them via your
{\tt H5VL\_class\_t} struct.

\subsection{Handling Optional Operations}

Handling optional operations has changed significantly in HDF5 1.13.0. In the
past, optional operations were specified using an integer {\tt opt\_type}
parameter. This proved to be a problem with pass-through connectors, though,
as it was possible to have {\tt opt\_type} clash if two connectors used
the same {\tt opt\_type} values.

The new scheme allows a connector to register an optional operation with
the library and receive a dynamically-allocated {\tt opt\_type} value
for the operation.

The following API calls can be used to manage the optional operations:
\begin{lstlisting}
herr_t H5VLregister_opt_operation(H5VL_subclass_t subcls, const char *op_name, int *op_val);
herr_t H5VLfind_opt_operation(H5VL_subclass_t subcls, const char *op_name, int *op_val);
herr_t H5VLunregister_opt_operation(H5VL_subclass_t subcls, const char *op_name);
\end{lstlisting}

The {\tt register} call is used to register an operation for a subclass (file, etc.)
and the {\tt opt\_type} parameter that the library assigned to the operation will be
returned via the {\tt opt\_val} parameter. This value can then be passed to one
of the subclass-specific API calls (listed below). If you need to find an existing
optional call's assigned {\tt opt\_type} value by name, you can use the {\tt find}
call.

One recommended way to handle optional calls is to register all the optional calls
at startup, saving the values in connector state, then use these cached values
in your optional calls. The assigned values should be unregistered using the
{\tt unregister} call when the connector shuts down.

Subclass-specific optional calls:
\begin{lstlisting}
herr_t H5VLattr_optional_op(const char *app_file, const char *app_func, unsigned app_line,
                            hid_t attr_id, H5VL_optional_args_t *args, hid_t dxpl_id, hid_t es_id);
herr_t H5VLdataset_optional_op(const char *app_file, const char *app_func, unsigned app_line,
                               hid_t dset_id, H5VL_optional_args_t *args, hid_t dxpl_id, hid_t es_id);
herr_t H5VLdatatype_optional_op(const char *app_file, const char *app_func, unsigned app_line,
                                hid_t type_id, H5VL_optional_args_t *args, hid_t dxpl_id, hid_t es_id);
herr_t H5VLfile_optional_op(const char *app_file, const char *app_func, unsigned app_line,
                            hid_t file_id, H5VL_optional_args_t *args, hid_t dxpl_id, hid_t es_id);
herr_t H5VLgroup_optional_op(const char *app_file, const char *app_func, unsigned app_line,
                             hid_t group_id, H5VL_optional_args_t *args, hid_t dxpl_id, hid_t es_id);
herr_t H5VLlink_optional_op(const char *app_file, const char *app_func, unsigned app_line,
                            hid_t loc_id, const char *name, hid_t lapl_id, H5VL_optional_args_t *args,
                            hid_t dxpl_id, hid_t es_id);
herr_t H5VLobject_optional_op(const char *app_file, const char *app_func, unsigned app_line,
                              hid_t loc_id, const char *name, hid_t lapl_id,
                              H5VL_optional_args_t *args, hid_t dxpl_id, hid_t es_id);
herr_t H5VLrequest_optional_op(void *req, hid_t connector_id, H5VL_optional_args_t *args);
\end{lstlisting}

\subsection{Testing Your Connector}

At the time of writing, some of the HDF5 library tests have been abstracted
out of the library with their native-file-format-only sections removed and
added to a VOL test suite available here:

\quad \quad \url{https://github.com/HDFGroup/vol-tests}

This is an evolving set of tests, so see the documentation in that repository
for instructions as to its use. You may want to clone and modify and/or
extend these tests for use with your own connector.

In the future, we plan to modify the HDF5 test suite that ships with the
library to use a future VOL capabilities flags scheme to selectively run
tests that a particular connector supports. As this is a large task, it may be
some time before that work is complete.

\subsection{Passthrough Connectors}

Coming Soon

\subsection{Asynchronous Operations}

Coming Soon


%%% Local Variables: 
%%% mode: latex
%%% TeX-master: t
%%% End: 
