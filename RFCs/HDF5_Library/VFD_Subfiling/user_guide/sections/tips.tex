\documentclass[../main.tex]{subfiles}
 
\begin{document}

\newpage

\section{Tips and best practices}
\label{sec:tips}

\begin{itemize}
\item While the default configuration for the Subfiling VFD has been found
to be generally performant across different machines, it is still a good
idea to initially test a range of Subfiling configurations to see if there
is a particular one that generally performs best on the machine of interest.

\item When using a parallel file system that stripes data across different
storage targets (e.g., Lustre, GPFS, etc.), it is generally a good idea to
choose a Subfiling stripe size that is a multiple of the size of the data
stripes. For the best performance, it is also generally recommended that
the HDF5 application use the \texttt{H5Pset\_alignment} API routine to set
the alignment of objects in the file to match the Subfiling stripe size or
to at least be a multiple of the size of the file system's data stripes.
However, setting the alignment of objects in the file can potentially waste
space and vastly increase the size of the resulting file, so caution may be
needed when trying to find a good balance with this suggestion.

\item If using the Subfiling VFD with a parallel HDF5 application that
pre-creates an HDF5 file on a single MPI rank or some other subset of
\texttt{MPI\_COMM\_WORLD}, it is necessary to adjust the "stripe count" (or
number of subfiles) value that the Subfiling VFD uses internally in order to
ensure that the proper number of subfiles gets generated when the logical HDF5
file is created. Without setting this value, the VFD will default to creating
one subfile per machine node, which will cause it to create only one subfile
in the case of pre-creating an HDF5 file on a single MPI rank. Note that as
pre-creating an HDF5 file on a subset of \texttt{MPI\_COMM\_WORLD} is expected
to be a less common operation when using the Subfiling VFD, an environment
variable for modifying this value has not been provided at this time and the
VFD will need to be configured on a File Access Property List for this use case.
Using the example in section \ref{ex:ex2} as a guide, one can configure the
stripe count value by setting the \texttt{stripe\_count} field in the
\texttt{H5FD\_subfiling\_params\_t} structure, which is part of the
\texttt{H5FD\_subfiling\_config\_t} structure that is passed to 
\texttt{H5Pset\_fapl\_subfiling}. For that particular example, one would
set the \texttt{subf\_config.shared\_cfg.stripe\_count} field to the desired
total number of subfiles.

\item While not always leading to an improvement in performance, it can
sometimes be useful to enable collective metadata writes and, if the application's
access pattern allows for it, collective metadata reads in a parallel HDF5
application through use of the \texttt{H5Pset\_coll\_metadata\_write} and
\texttt{H5Pset\_all\_coll\_metadata\_ops} API routines.

\item When measuring the performance of the Subfiling feature, be aware that
results may sometimes be artificially high due to file system caching.
Since all I/O to a particular subfile is directed to a single I/O Concentrator
on a machine node, there are many opportunities for data from subfiles to
be served from or written to file system caching layers.
\end{itemize}

\end{document}