\documentclass[../main.tex]{subfiles}

\begin{document}

\section{Reference Manual}

The section contains a list of the C APIs related to the Subfiling feature. The Fortran APIs have similar signatures and can be found online in the HDF5 Reference Manual.

\label{apdx:ref_manual}

%%%%%%%%%%%%%%%%%%%%%%%%%%%%%%%%%%%%%%%%%%%%%%%%%%%%%%%%%%%%%%%%%%%%%%%%%%%%%%
% Function Name
\subsection{H5Pset\_fapl\_subfiling}
\label{ref:h5p_set_fapl_subfiling}

% Function
\paragraph{Synopsis:}
\begin{flushleft}%
\begin{minted}[breaklines=true,fontsize=\small]{hdf5-c-lexer.py:HDF5CLexer -x}
herr_t H5Pset_fapl_subfiling(hid_t fapl_id,
                             const H5FD_subfiling_config_t * vfd_config);
\end{minted}
\end{flushleft}%

% Purpose
\paragraph{Purpose:}
\begin{flushleft}%
Sets the Subfiling \Gls{VFD} on a \Gls{FAPL}.
\end{flushleft}%

% Description
\paragraph{Description:}
\begin{flushleft}%
\texttt{H5Pset\_fapl\_subfiling} modifies the given \Gls{FAPL} to use the
Subfiling \Gls{VFD}. Configuration for the File Driver is provided by a
pointer to a \texttt{H5FD\_subfiling\_config\_t} structure (\ref{ref:h5fd_subfiling_config_t}).
MPI must have been initialized within the HDF5 application using \texttt{MPI\_Init\_thread}
with the \texttt{MPI\_THREAD\_MULTIPLE} flag prior to calling this routine. Any
special MPI Communicator and MPI Info parameters must also have been set on
the given FAPL using the \texttt{H5P\_set\_mpi\_params} routine prior to calling
this routine.
\end{flushleft}%

% Parameters
\paragraph{Parameters:}
\begin{flushleft}%
 \begin{tabular}{lp{0.4\textwidth}}%
   \texttt{hid\_t fapl\_id} & IN: \Gls{FAPL} ID \\
   \texttt{const H5FD\_subfiling\_config\_t * vfd\_config} & IN: Pointer to Subfiling VFD configuration structure. May be NULL. \\
 \end{tabular}%
\end{flushleft}%

% Return
\paragraph{Returns:}
\begin{flushleft}%
Returns a non-negative value if successful; otherwise returns a negative value.
\end{flushleft}%

\newpage

%%%%%%%%%%%%%%%%%%%%%%%%%%%%%%%%%%%%%%%%%%%%%%%%%%%%%%%%%%%%%%%%%%%%%%%%%%%%%%
% Function Name
\subsection{H5Pget\_fapl\_subfiling}
\label{ref:h5p_get_fapl_subfiling}

% Function
\paragraph{Synopsis:}
\begin{flushleft}%
\begin{minted}[breaklines=true,fontsize=\small]{hdf5-c-lexer.py:HDF5CLexer -x}
herr_t H5Pget_fapl_subfiling(hid_t fapl_id,
                             H5FD_subfiling_config_t * config_out);
\end{minted}
\end{flushleft}%

% Purpose
\paragraph{Purpose:}
\begin{flushleft}%
Retrieves the Subfiling \Gls{VFD} configuration set on the given \Gls{FAPL}.
\end{flushleft}%

% Description
\paragraph{Description:}
\begin{flushleft}%
\texttt{H5Pget\_fapl\_subfiling} returns the Subfiling configuration set on
the given \Gls{FAPL} through the provided \texttt{H5FD\_subfiling\_config\_t}
structure pointer. If no such configuration has been set on the given FAPL,
a default Subfiling configuration is returned. An HDF5 application may use
this functionality to configure the Subfiling VFD by obtaining a default
configuration through calling \texttt{H5Pget\_fapl\_subfiling} on a newly-created
\Gls{FAPL}, adjusting the default values and then calling \texttt{H5Pset\_fapl\_subfiling}
with the configured \texttt{H5FD\_subfiling\_config\_t} structure.

\textbf{NOTE:} \texttt{H5Pget\_fapl\_subfiling} only returns the Subfiling
configuration parameters as set by calling \texttt{H5Pset\_fapl\_subfiling}.
Any configuration parameters set through the use of environment variables
will not be reflected in the configuration returned by this routine, so an
application may need to check those environment variables to get accurate
values for the Subfiling VFD configuration.
\end{flushleft}%

% Parameters
\paragraph{Parameters:}
\begin{flushleft}%
 \begin{tabular}{lp{0.4\textwidth}}%
   \texttt{hid\_t fapl\_id} & IN: \Gls{FAPL} ID \\
   \texttt{H5FD\_subfiling\_config\_t * config\_out} & OUT: Pointer to
   Subfiling VFD configuration structure \ through which the Subfiling VFD configuration
   information will be returned. \\
 \end{tabular}%
\end{flushleft}%

% Return
\paragraph{Returns:}
\begin{flushleft}%
Returns a non-negative value if successful; otherwise returns a negative value.
\end{flushleft}%

\newpage

%%%%%%%%%%%%%%%%%%%%%%%%%%%%%%%%%%%%%%%%%%%%%%%%%%%%%%%%%%%%%%%%%%%%%%%%%%%%%%
% Function Name
\subsection{H5Pset\_fapl\_ioc}
\label{ref:h5p_set_fapl_ioc}

% Function
\paragraph{Synopsis:}
\begin{flushleft}%
\begin{minted}[breaklines=true,fontsize=\small]{hdf5-c-lexer.py:HDF5CLexer -x}
herr_t H5Pset_fapl_ioc(hid_t fapl_id,
                       H5FD_ioc_config_t * vfd_config);
\end{minted}
\end{flushleft}%

% Purpose
\paragraph{Purpose:}
\begin{flushleft}%
Sets the "IOC" I/O Concentrator \Gls{VFD} on a \Gls{FAPL}.
\end{flushleft}%

% Description
\paragraph{Description:}
\begin{flushleft}%
\texttt{H5Pset\_fapl\_ioc} modifies the given \Gls{FAPL} to use the
"IOC" I/O Concentrator \Gls{VFD}. Configuration for the File Driver
is provided by a pointer to a \texttt{H5FD\_ioc\_config\_t} structure
(\ref{ref:h5fd_ioc_config_t}).
\end{flushleft}%

% Parameters
\paragraph{Parameters:}
\begin{flushleft}%
 \begin{tabular}{lp{0.5\textwidth}}%
   \texttt{hid\_t fapl\_id} & IN: \Gls{FAPL} ID \\
   \texttt{H5FD\_ioc\_config\_t * vfd\_config} & IN: Pointer to
   "IOC" I/O Concentrator VFD configuration structure. May be NULL. \\
 \end{tabular}%
\end{flushleft}%

% Return
\paragraph{Returns:}
\begin{flushleft}%
Returns a non-negative value if successful; otherwise returns a negative value.
\end{flushleft}%

\newpage

%%%%%%%%%%%%%%%%%%%%%%%%%%%%%%%%%%%%%%%%%%%%%%%%%%%%%%%%%%%%%%%%%%%%%%%%%%%%%%
% Function Name
\subsection{H5Pget\_fapl\_ioc}
\label{ref:h5p_get_fapl_ioc}

% Function
\paragraph{Synopsis:}
\begin{flushleft}%
\begin{minted}[breaklines=true,fontsize=\small]{hdf5-c-lexer.py:HDF5CLexer -x}
herr_t H5Pget_fapl_ioc(hid_t fapl_id,
                       H5FD_ioc_config_t * config_out);
\end{minted}
\end{flushleft}%

% Purpose
\paragraph{Purpose:}
\begin{flushleft}%
Retrieves the "IOC" I/O Concentrator \Gls{VFD} configuration set on the given \Gls{FAPL}.
\end{flushleft}%

% Description
\paragraph{Description:}
\begin{flushleft}%
\texttt{H5Pget\_fapl\_ioc} returns the "IOC" I/O Concentrator configuration set
on the given \Gls{FAPL} through the provided \texttt{H5FD\_ioc\_config\_t}
structure pointer. If no such configuration has been set on the given FAPL,
a default "IOC" I/O Concentrator configuration is returned. An HDF5 application
may use this functionality to configure the "IOC" I/O Concentrator VFD by obtaining
a default configuration through calling \texttt{H5Pget\_fapl\_ioc} on a newly-created
\Gls{FAPL}, adjusting the default values and then calling \texttt{H5Pset\_fapl\_ioc}
with the configured \texttt{H5FD\_ioc\_config\_t} structure.

\textbf{NOTE:} \texttt{H5Pget\_fapl\_ioc} only returns the "IOC" I/O Concentrator
configuration parameters as set by calling \texttt{H5Pset\_fapl\_ioc}. Any
configuration parameters set through the use of environment variables will not
be reflected in the configuration returned by this routine, so an application may
need to check those environment variables to get accurate values for the "IOC" I/O
Concentrator VFD configuration.
\end{flushleft}%

% Parameters
\paragraph{Parameters:}
\begin{flushleft}%
 \begin{tabular}{lp{0.5\textwidth}}%
   \texttt{hid\_t fapl\_id} & IN: \Gls{FAPL} ID \\
   \texttt{H5FD\_ioc\_config\_t * config\_out} & OUT: Pointer to
   "IOC" I/O Concentrator VFD configuration structure through which the "IOC" I/O Concentrator
   VFD configuration information will be returned. \\
 \end{tabular}%
\end{flushleft}%

% Return
\paragraph{Returns:}
\begin{flushleft}%
Returns a non-negative value if successful; otherwise returns a negative value.
\end{flushleft}%

\newpage

%%%%%%%%%%%%%%%%%%%%%%%%%%%%%%%%%%%%%%%%%%%%%%%%%%%%%%%%%%%%%%%%%%%%%%%%%%%%%%
\subsection{H5FD\_subfiling\_config\_t}
\label{ref:h5fd_subfiling_config_t}

\paragraph{Type declaration:}
\begin{flushleft}%
\begin{minted}[breaklines=true,fontsize=\small]{hdf5-c-lexer.py:HDF5CLexer -x}
typedef struct H5FD_subfiling_config_t {
    uint32_t magic;
    uint32_t version;
    hid_t    ioc_fapl_id;
    hbool_t  require_ioc;
    H5FD_subfiling_params_t shared_cfg;
} H5FD_subfiling_config_t;
\end{minted}
\end{flushleft}%

\paragraph{Purpose:}
\begin{flushleft}%
Subfiling configuration structure provided to the \texttt{H5Pset\_fapl\_subfiling} API routine
\end{flushleft}%

\paragraph{Description of fields:}
\begin{flushleft}%
\texttt{uint32\_t magic} - This field is a "magic" number used for Subfiling configuration
verification and must currently always be set to the value represented by the
\texttt{H5FD\_SUBFILING\_FAPL\_MAGIC} macro.

\texttt{uint32\_t version} - This field is the version number of the Subfiling configuration
structure and should currently always be set to the value represented by the
\texttt{H5FD\_SUBFILING\_CURR\_FAPL\_VERSION} macro.

\texttt{hid\_t ioc\_fapl\_id} - This field is the ID of the \Gls{FAPL} that the Subfiling VFD
will use for accessing subfiles and must be setup with an I/O Concentrator VFD in order for
the Subfiling VFD to operate properly. Currently, this FAPL will be setup with the "IOC" VFD
by default and shouldn't be modified, but future work may allow other I/O Concentrator VFDs to
be set on this FAPL instead of the default reference implementation VFD.

\texttt{hbool\_t require\_ioc} - This field determines whether the Subfiling VFD should require
an underlying I/O Concentrator VFD and should currently always be set to \texttt{true}.

\texttt{H5FD\_subfiling\_params\_t shared\_cfg} - This field is a structure that contains the
various parameters for the Subfiling VFD, such as the data stripe size, number of subfiles to
be used, etc. Refer to \ref{ref:h5fd_subfiling_params_t} for information about its fields.
\end{flushleft}%

\newpage

%%%%%%%%%%%%%%%%%%%%%%%%%%%%%%%%%%%%%%%%%%%%%%%%%%%%%%%%%%%%%%%%%%%%%%%%%%%%%%
\subsection{H5FD\_subfiling\_params\_t}
\label{ref:h5fd_subfiling_params_t}

\paragraph{Type declaration:}
\begin{flushleft}%
\begin{minted}[breaklines=true,fontsize=\small]{hdf5-c-lexer.py:HDF5CLexer -x}
typedef struct H5FD_subfiling_params_t {
    H5FD_subfiling_ioc_select_t ioc_selection;
    int64_t                     stripe_size;
    int32_t                     stripe_count;
} H5FD_subfiling_params_t;
\end{minted}
\end{flushleft}%

\paragraph{Purpose:}
\begin{flushleft}%
Structure containing various parameters for the Subfiling VFD
\end{flushleft}%

\paragraph{Description of fields:}
\begin{flushleft}%
\texttt{H5FD\_subfiling\_ioc\_select\_t ioc\_selection} - This field is an enumeration that
determines how the Subfiling VFD chooses MPI ranks, threads, etc. as I/O Concentrators. Refer
to \ref{ref:h5fd_subfiling_ioc_select_t} for information about its defined values.

\texttt{int64\_t stripe\_size} - This field contains the size (in bytes) of data stripes
written to subfiles. The default setting for this parameter is a 32MiB stripe size. The value
set must either be the value defined by the \texttt{H5FD\_SUBFILING\_DEFAULT\_STRIPE\_SIZE}
macro, or an integer value $> 0$.

\texttt{int32\_t stripe\_count} - This field contains the target number of subfiles to use.
It is mostly useful when pre-creating an HDF5 file in a parallel application within some
subset of \texttt{MPI\_COMM\_WORLD} (see section \ref{sec:tips}). The default setting for
this parameter is set such that one subfile is used per machine node. The value set must
either be the value defined by the \texttt{H5FD\_SUBFILING\_DEFAULT\_STRIPE\_COUNT} macro,
or an integer value $> 0$.
\end{flushleft}%

\newpage

%%%%%%%%%%%%%%%%%%%%%%%%%%%%%%%%%%%%%%%%%%%%%%%%%%%%%%%%%%%%%%%%%%%%%%%%%%%%%%
\subsection{H5FD\_subfiling\_ioc\_select\_t}
\label{ref:h5fd_subfiling_ioc_select_t}

\paragraph{Type declaration:}
\begin{flushleft}%
\begin{minted}[breaklines=true,fontsize=\small]{hdf5-c-lexer.py:HDF5CLexer -x}
typedef enum {
    SELECT_IOC_ONE_PER_NODE = 0,
    SELECT_IOC_EVERY_NTH_RANK,
    SELECT_IOC_WITH_CONFIG,
    SELECT_IOC_TOTAL,
    ioc_selection_options
} H5FD_subfiling_ioc_select_t;
\end{minted}
\end{flushleft}%

\paragraph{Purpose:}
\begin{flushleft}%
Enumeration containing the values for different methods of selection I/O Concentrators
\end{flushleft}%

\paragraph{Description of values:}
\begin{flushleft}%
\texttt{SELECT\_IOC\_ONE\_PER\_NODE} - The default I/O Concentrator selection method. I/O
Concentrators will be selected such that only one is assigned per machine node. If this
selection method is specified, the \texttt{H5FD\_SUBFILING\_IOC\_PER\_NODE} environment
variable can be used to adjust the number of I/O Concentrators assigned per machine node.

\texttt{SELECT\_IOC\_EVERY\_NTH\_RANK} - I/O Concentrators will be selected such that an
initial I/O Concentrator is selected from the available candidates and then each following
I/O Concentrator is selected by applying a stride value \texttt{N}. For example, if I/O
Concentrators are MPI ranks, MPI rank 0 will be the first I/O Concentrator and then the
next I/O Concentrator will be MPI rank \texttt{N} and so on. The default value for
\texttt{N} is 1 and can be adjusted with the \texttt{H5FD\_SUBFILING\_IOC\_SELECTION\_CRITERIA}
environment variable.

\texttt{SELECT\_IOC\_WITH\_CONFIG} - This method of I/O Concentrator selection is currently
unsupported.

\texttt{SELECT\_IOC\_TOTAL} - I/O Concentrators will be selected such that a total number
\texttt{N} of I/O Concentrators are assigned. A fixed stride value is applied when selecting
I/O Concentrators so that they are evenly spaced across the set of available candidates. The
default value for \texttt{N} is one and can be adjusted with the
\texttt{H5FD\_SUBFILING\_IOC\_SELECTION\_CRITERIA} environment variable.

\texttt{ioc\_selection\_options} - This is a sentinel value and should not be used.
\end{flushleft}%

\newpage

%%%%%%%%%%%%%%%%%%%%%%%%%%%%%%%%%%%%%%%%%%%%%%%%%%%%%%%%%%%%%%%%%%%%%%%%%%%%%%
\subsection{H5FD\_ioc\_config\_t}
\label{ref:h5fd_ioc_config_t}

\paragraph{Type declaration:}
\begin{flushleft}%
\begin{minted}[breaklines=true,fontsize=\small]{hdf5-c-lexer.py:HDF5CLexer -x}
typedef struct H5FD_ioc_config_t {
    uint32_t magic;
    uint32_t version;
    int32_t  thread_pool_size;
} H5FD_ioc_config_t;
\end{minted}
\end{flushleft}%

\paragraph{Purpose:}
\begin{flushleft}%
"IOC" I/O Concentrator VFD configuration structure provided to the \texttt{H5Pset\_fapl\_ioc}
API routine
\end{flushleft}%

\paragraph{Description of fields:}
\begin{flushleft}%
\texttt{uint32\_t magic} - This field is a "magic" number used for "IOC" I/O Concentrator VFD
configuration verification and must currently always be set to the value represented by the
\texttt{H5FD\_IOC\_FAPL\_MAGIC} macro.

\texttt{uint32\_t version} - This field is the version number of the "IOC" I/O Concentrator VFD
configuration structure and should currently always be set to the value represented by the
\texttt{H5FD\_IOC\_CURR\_FAPL\_VERSION} macro.

\texttt{int32\_t thread\_pool\_size} - This field contains the number of worker threads to
use per I/O Concentrator. The default setting for this parameter is 4 worker threads per I/O
Concentrator. The value set must either be the value defined by the \texttt{H5FD\_IOC\_DEFAULT\_THREAD\_POOL\_SIZE}
macro, or an integer value $> 0$.
\end{flushleft}%

\newpage

%%%%%%%%%%%%%%%%%%%%%%%%%%%%%%%%%%%%%%%%%%%%%%%%%%%%%%%%%%%%%%%%%%%%%%%%%%%%%%
\subsection{H5FD\_SUBFILING\_FILENAME\_TEMPLATE}
\label{ref:h5fd_subfiling_filename_template}

\paragraph{Macro declaration:}
\begin{flushleft}%
\begin{minted}[breaklines=true,fontsize=\small]{hdf5-c-lexer.py:HDF5CLexer -x}
#define H5FD_SUBFILING_FILENAME_TEMPLATE "%s.subfile_%" PRIu64 "_%0*d_of_%d"
\end{minted}
\end{flushleft}%

\paragraph{Description:}
\begin{flushleft}%
This is a convenience macro for external tooling that defines the printf-style
format template used to generate the filenames for subfiles. Its format specifiers
are as follows:

\begin{verbatim}
%s      -> base filename, e.g. "file.h5"
%PRIu64 -> file inode, e.g. 11273556
%0*d    -> number (starting at 1) signifying the Nth (out of total
           number of subfiles) subfile. Zero-padded according
           to the number of digits in the number of subfiles
           (calculated by log10(num_subfiles) + 1)
%d      -> number of subfiles
\end{verbatim}

yielding filenames such as:

\begin{verbatim}
file.h5.subfile_11273556_01_of_10
file.h5.subfile_11273556_02_of_10
file.h5.subfile_11273556_10_of_10
\end{verbatim}
\end{flushleft}%

\newpage

%%%%%%%%%%%%%%%%%%%%%%%%%%%%%%%%%%%%%%%%%%%%%%%%%%%%%%%%%%%%%%%%%%%%%%%%%%%%%%
\subsection{H5FD\_SUBFILING\_CONFIG\_FILENAME\_TEMPLATE}
\label{ref:h5fd_subfiling_config_filename_template}

\paragraph{Macro declaration:}
\begin{flushleft}%
\begin{minted}[breaklines=true,fontsize=\small]{hdf5-c-lexer.py:HDF5CLexer -x}
#define H5FD_SUBFILING_CONFIG_FILENAME_TEMPLATE "%s.subfile_%" PRIu64 ".config"
\end{minted}
\end{flushleft}%

\paragraph{Description:}
\begin{flushleft}%
This is a convenience macro for external tooling that defines the printf-style
format template used to generate a Subfiling configuration file. Its format specifiers
are as follows:

\begin{verbatim}
%s      -> base filename, e.g. "file.h5"
%PRIu64 -> file inode, e.g. 11273556
\end{verbatim}

yielding a filename such as:

\begin{verbatim}
file.h5.subfile_11273556.config
\end{verbatim}
\end{flushleft}%

\newpage

\end{document}
