\documentclass[letterpaper,hyper]{THG_RFC}

% Fonts
\usepackage{lmodern} % times / lmodern / mathpazo / palatino
\usepackage[scaled=.95]{helvet}
\usepackage{courier}

% Path to figures, plots
\graphicspath{{./pics/}{./plots/}}

% Code snippets
\usepackage{listings}
\def\lstsetc{\lstset{language=C,
  numbers=left,
  xleftmargin=20pt,
  numberstyle=\tiny\color{gray},
  stepnumber=1,
  showspaces=false, 
  showstringspaces=false,
  breaklines=true,
  basicstyle=\footnotesize\ttfamily,
  stringstyle=\itshape,
  commentstyle=\itshape\bfseries,
  morekeywords={RFC, template}
  }
}

% Title, author, etc
\title{Title}
\author{Author1}
\author{Author2}
\date{July 29, 2014}
\rfcversion{2014-07-29.v1}
\revision{July 29, 2014}{Version 1 circulated for comment within The HDF Group.}

%% Start the document
\begin{document}

%% Title
\maketitle

%% Abstract
\begin{abstract}
First, have an abstract. Briefly describe the reason for the RFC and the recommendation it makes. Provide just enough information to let the reader decide if they should keep reading, or if RFC is of no interest to them.
\end{abstract}

\section{Introduction}
Every RFC should have an introduction. Tell the reader about the problem this RFC is designed to address, or the enhancement it will provide. The introduction should give the reader enough background that they can decide if they want to keep reading.

\section{Approach / Motivation / Etc.}
Typically the first section after the Introduction will go into motivation in more depth, talk about approach or whatever you want. There is no requirement for the actual Section name. It may have subsections. Do what makes sense for your RFC.

\section{Standard Parts of the Template}

\section{Recommendation}

%% Close
\section*{Revision History}
\makerevisions

%% References
%\bibliographystyle{ieeetr}
%\bibliography{}

\end{document}
