\documentclass[letterpaper,hyper]{THG_RFC}

% Fonts
\usepackage{ifluatex}
\ifluatex
  % Font
  \usepackage{luatextra, xunicode, unicode-math}
  \defaultfontfeatures{Ligatures=TeX}
  \setmainfont[
    UprightFont={*},
    BoldFont={*-Bold},
    ItalicFont={*-Italic},
    BoldItalicFont={*-BoldItalic},
    SlantedFont={*-Italic},
    BoldSlantedFont={*-BoldItalic}
  ]{CMU Serif}
  \setsansfont[
    UprightFont={*},
    BoldFont={*-Bold},
    ItalicFont={*-Oblique},
    BoldItalicFont={*-BoldOblique}
  ]{CMU Sans Serif}
  \setmathfont[math-style=ISO, bold-style=ISO]{Asana Math}
  \setmonofont[
    UprightFont={*-Regular},
    BoldFont={*-Bold},
    ItalicFont={*-Italic},
    BoldItalicFont={*-BoldItalic}
  ]{CMU Typewriter}
\else
  \usepackage{times} % times / lmodern / mathpazo / palatino
  \usepackage[scaled=.95]{helvet}
  \usepackage{courier}
\fi

% Path to figures, plots
\graphicspath{{./pics/}{./plots/}}

%% TikZ
\usepackage{tikz}
\usetikzlibrary{patterns, shapes, decorations.pathreplacing}
\usepackage{gnuplot-lua-tikz}

% SI units
%\usepackage{textcomp}
\usepackage[binary-units]{siunitx}
\sisetup{per-mode = symbol}
\providecommand{\gbps}[1]{\SI{#1}{\giga\byte\per\second}}
\providecommand{\mbps}[1]{\SI{#1}{\mega\byte\per\second}}
\providecommand{\gb}[1]{\SI{#1}{\giga\byte}}
\providecommand{\mb}[1]{\SI{#1}{\mega\byte}}
\providecommand{\kb}[1]{\SI{#1}{\kilo\byte}}


% Code snippets
\usepackage{listings}
\def\lstsetc{\lstset{language=C,
  numbers=left,
  xleftmargin=20pt,
  numberstyle=\tiny\color{gray},
  stepnumber=1,
  showspaces=false, 
  showstringspaces=false,
  breaklines=true,
  basicstyle=\small\ttfamily,
  stringstyle=\itshape,
  commentstyle=\itshape\bfseries,
  morekeywords={main, size_t, malloc, free,
    hsize_t, hid_t, herr_t,
    H5Fcreate, H5Fclose,
    H5Dcreate, H5Dopen, H5Dclose, H5Dwrite, H5Dread,
    H5Screate_simple, H5Sclose, H5Sget_select_npoints
    }
  }
}

% Title, author, etc
\title{New Datatypes}
\author{Jerome Soumagne}
\author{Quincey Koziol / Elena Pourmal}
\date{June 11, 2015}
\rfcversion{2015-04-29.v3}
\revision{April 29, 2015}{Version 1 circulated for comment within The HDF Group.}
\revision{June 10, 2015}{Version 2 circulated for comment within The HDF Group.}
\revision{June 11, 2015}{Version 3 circulated for comment within The HDF Group.}

%% Start the document
\begin{document}

%% Add Attribute reference RFC

%% Title
\maketitle

%% Abstract
\begin{abstract}
The C99 standard introduced support for boolean and complex types.
Boolean and complex types are also supported in C++, Fortran, Java, Python.
This RFC describes how support for these types can be added within the HDF5 library.
\end{abstract}

\section{Introduction}
The HDF5 library defines already datatypes for all common types, i.e., char,
short, int, long, etc, as well as their specific encoded variation, for instance
all standard types, as well as little/big endian encoding, etc.
However, new datatypes were requested by HDF5 users (ESDIS, LLNL, etc) who
work with scientific applications to support both \textit{boolean} and
\textit{complex} types. Boolean types should be seen as a distinct conceptual
type, not as another integer type. An important
aspect of having a boolean datatype, despite the convenient mapping that it can
provide between C/C++ codes that use boolean concepts, is to no longer require
the use of integers for storing boolean data and therefore reduce storage needs. 
Regarding complex types, they are already natively defined by the C standard,
therefore not only for convenience but also for performance reasons would it
makes sense for the HDF5 library to provide a native datatype so that users
are no longer forced to either use separate datasets or create a
compound type (which would in this case be composed of real/imaginary fields)
in order to store and retrieve data. As these types were introduced by the
C99 standard, this set of datatypes can be defined as new HDF5 native types.

\section{New Datatypes}

The following types are considered: \texttt{\_Bool}, \texttt{float \_Complex},
\texttt{double \_Complex}, \texttt{long double \_Complex}.
Next sections are a reminder of the rules that the C standard~\cite{c11} defines
for these types.

\subsection{\texttt{\_Bool}}

An object declared as type \texttt{\_Bool} is large enough to store the
values 0 and 1.
While the number of bits in a \texttt{\_Bool} object is at least
\texttt{CHAR\_BIT}, the width (number of sign and value bits) of a
\texttt{\_Bool} may be just 1 bit.

\paragraph{Conversion rules:}
When any scalar value is converted to \texttt{\_Bool}, the result is 0 if the
value compares equal to 0; otherwise, the result is 1.

\subsection{\texttt{\_Complex}}

An object declared as type complex is composed of two parts: real and imaginary,
the real part is always before the imaginary part.
Values of complex types are equal if and only if both their real parts are equal
and also their imaginary parts are equal.
The integer constant 1, \texttt{\_\_STDC\_IEC\_559\_COMPLEX\_\_}, intended to
indicate adherence to the IEC 60559 compatible complex arithmetic specifications.
The integer constant 1, \texttt{\_\_STDC\_NO\_COMPLEX\_\_}, intended to indicate
that the implementation does not support complex types or the \texttt{<complex.h>}
header.

{\lstsetc
\begin{lstlisting}
#include<stdio.h>
#include<complex.h>

union Example
{
  float f[2];
  float complex c;
};

int main()
{
  union Example example;
  example.c = 1.0f + 0.0f*I;
  printf("First element of float: %.4f\n", example.f[0]);
  printf("Second element of float: %.4f\n", example.f[1]);
  printf("Real part of complex: %.4f\n", __real__(example.c));
  printf("Imaginary part of complex: %.4fi\n", __imag__(example.c));
  return 0;
}
\end{lstlisting}
}

As mentioned in~\cite{ibm2014}, on both big endian and little endian, the first
element of f is 1.0000 and the second element of f is 0.0000. The real part of
the complex number c is 1.0000 and the imaginary part is 0.0000i.
\begin{verbatim}
First element of float: 1.0000
Second element of float: 0.0000
Real part of complex: 1.0000
Imaginary part of complex: 0.0000i
\end{verbatim}
The byte order of the elements of a floating-point and a complex number is
different between big endian and little endian, but the element order is the
same.

\paragraph{Conversion rules:}
\begin{table}[ht]\footnotesize
\caption{Conversion to/from \texttt{\_Complex} (\texttt{a+b*i}).}
\label{tab:complex_convert}
\begin{tabular}{lccc} \toprule
Types     & Complex                    & Real             & Imaginary          \\
\midrule
Complex   & \texttt{conv(a)+conv(b)*i} & \texttt{conv(a)} & \texttt{conv(b)*i} \\
Real      & \texttt{conv(a)+0*i}       & -                & \texttt{0*i}       \\
Imaginary & \texttt{0+conv(b)*i}       & \texttt{0}       & -                  \\
\bottomrule
\end{tabular}
\end{table}
As summarized in~\TableRef{tab:complex_convert},
when a value of complex type is converted to another complex type, both the real
and imaginary parts follow the conversion rules for the corresponding real types.
When a value of real type is converted to a complex type, the real part of the
complex result value is determined by the rules of conversion to the
corresponding real type and the imaginary part of the complex result value is a
positive zero or an unsigned zero.
When a value of complex type is converted to a real type, the imaginary part of
the complex value is discarded and the value of the real part is converted
according to the conversion rules for the corresponding real type.
When a value of imaginary type is converted to a real type other than
\texttt{\_Bool}, the result is a positive zero.
When a value of real type is converted to an imaginary type, the result is a
positive imaginary zero.
When a value of imaginary type is converted to a complex type, the real part of
the complex result value is a positive zero and the imaginary part of the
complex result value is determined by the conversion rules for the
corresponding real types.
When a value of complex type is converted to an imaginary type, the real part
of the complex value is discarded and the value of the imaginary part is
converted according to the conversion rules for the corresponding real types.

\section{HDF5 Implementation}

Implementing these new types requires modifications of multiple HDF5 components:
the core library, the high-level library, the Fortran library, the C++ library,
the tools. This implies modification of the corresponding tests that support
these components, as well as of the file format.

\subsection{Core Library Support}
As already described in~\cite{lu2012}, implementation within the HDF5 library
of a new datatype requires the following changes:
\begin{enumerate}
\item Detect the size of the new type in configure/CMake along with other native datatypes. After the configuration, a new macro called \texttt{H5\_SIZEOF\_XXX}
is defined as in \texttt{H5pubconf.h} in the src/ directory under the build
directory. If the compiler does not support this datatype, the value of
\texttt{H5\_SIZEOF\_XXX} would be 0.
\item Add the new type for detection in \texttt{src/H5detect.c}. After compiling
\texttt{H5detect.c} and running \texttt{H5detect}, the properties of the type is
generated in \texttt{H5Tinit.c}. 
\item Declare the variables related to the new type in the top of \texttt{H5T.c}.
\item Release the variable in \texttt{H5T\_term\_interface} of \texttt{H5T.c}.
\item Make the datatype ID public by defining a macro in \texttt{H5Tpublic.h}.
\item Declare the variables for alignments in \texttt{H5Tpkg.h}.
\item Add a printout for the new type in \texttt{H5\_trace} of \texttt{H5trace.c}
along with other types.
\item Add the prototypes of hard conversion functions for this new datatype in
\texttt{H5Tpkg.h}. There should be functions between the new datatype and all
other native integer types and all native floating-point number types.
\item Add the definitions of the hard conversion functions for this new datatype
in \texttt{H5Tconv.c}.
\item Add test cases for the new type in the test suite, especially the data
conversion test \texttt{dt\_arith.c}.
\end{enumerate}

\subsubsection{Boolean Type}
\texttt{H5T\_NATIVE\_BOOL} will be detected and its
size\footnote{By default, when compiling with GCC, \texttt{sizeof(bool)}
is 4 when compiling for Darwin/PowerPC and 1 when compiling for Darwin/x86.}
defined accordingly.
Adding support for the type within the core library should be straightforward.
Conversion routines will follow rules already defined by the C standard.

%\begin{table}[ht]\footnotesize
%\caption{Conversion to/from \texttt{H5T\_NATIVE\_BOOL}.}
%\label{tab:bool_convert}
%\begin{tabular}{lccccccccc} \toprule
%Types & \texttt{\_Bool} &
%\texttt{char} & \texttt{short} & \texttt{int} & \texttt{long} & \texttt{llong}
%& \texttt{float} & \texttt{double} & \texttt{ldouble}\\
%\midrule
%\texttt{\_Bool}  & $2^1 - 1$ & $2^1 - 1$ & $2^1 - 1$ & $2^1 - 1$ & $2^1 - 1$ & $2^1 - 1$ & $2^1 - 1$ & $2^1 - 1$ & $2^1 - 1$ \\
%\texttt{char}    & X & X & X & X & X & X & X & X & X \\
%\texttt{short}   & X & X & X & X & X & X & X & X & X \\
%\texttt{int}     & X & X & X & X & X & X & X & X & X \\
%\texttt{long}    & X & X & X & X & X & X & X & X & X \\
%\texttt{llong}   & X & X & X & X & X & X & X & X & X \\
%\texttt{float}   & X & X & X & X & X & X & X & X & X \\
%\texttt{double}  & X & X & X & X & X & X & X & X & X \\
%\texttt{ldouble} & X & X & X & X & X & X & X & X & X \\
%\bottomrule
%\end{tabular}
%end{table}

\paragraph{HDF5 boolean type:} It is also worth noting that HDF5 used to define
a boolean type, referred to as \texttt{hbool\_t}. This boolean type is
currently defined as an integer, therefore for consistency it will now be
defined as \texttt{\_Bool} when possible.

\subsubsection{Complex Types}
\texttt{H5T\_NATIVE\_FLOAT\_COMPLEX}, \texttt{H5T\_NATIVE\_DOUBLE\_COMPLEX},
and \texttt{H5T\_NATIVE\_LDOUBLE\_COMPLEX} will be detected and their size
defined accordingly.
Adding support for this type may require some more work as one may need to
access real and imaginary parts within the functions that are already defined.
Conversion routines will follow rules already defined by the C standard.

\subsection{High-level Support}
The \texttt{H5LT} API makes minimal use of datatypes and support for these
additional datatypes will be added (straightforward).
High-level read or write make use of datatypes, they will need to be tested
using these new types.

\subsection{Fortran Support}
ISO C bindings are used to provide us with a standard mechanism to pass data
between Fortran and C. Fortran \texttt{LOGICAL} can be mapped to the C
\texttt{\_Bool} type by using a special integer kind (\texttt{C\_BOOL}).

\paragraph{Note.}
We will leave Fortran HL out for now.

\subsection{C++ Support}
C++ supports \texttt{bool} types, as well as complex numbers, which are available
through the standard library. 

\subsection{Python Support}
Python supports boolean values \texttt{True} and \texttt{False}, as well as
complex types. Therefore these new datatypes make perfect sense for Python
users.

\subsection{Java Support}
Java provides boolean types. Complex numbers are defined through classes.

\subsection{Tools Support}
The following list of tools must be adapted: \texttt{h5dump}, \texttt{h5ls},
\texttt{h5diff}, \texttt{h5import}, \texttt{h5repack}, \texttt{h5debug}. 

\section{Additional Questions}

\section*{Revision History}
\makerevisions

%% References
\bibliographystyle{ieeetr}
\bibliography{new_datatypes}

\end{document}
