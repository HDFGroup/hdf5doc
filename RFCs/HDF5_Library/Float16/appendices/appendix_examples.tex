\documentclass[../HDF5_RFC.tex]{subfiles}
 
\begin{document}

% Start each appendix on a new page
\newpage

\section{Appendix: Examples}
\label{apdx:examples}

The following are examples of what using these new datatypes should look like once
support has been implemented in HDF5. Note that these examples are subject to change,
depending on the final decisions made when implementing support for these new datatypes.
Once support is implemented, relevant examples will be added to HDF5 to serve as a
better guide.

\textbf{Create a dataset with a binary16 little-endian 16-bit floating-point datatype}

\begin{minted}[custom lexer]{hdf5-c-lexer.py:HDF5CLexer}
#include "hdf5.h"

int
main(int argc, char **argv)
{
    hsize_t dims[2] = { 10, 10 };
    hid_t file_id   = H5I_INVALID_HID;
    hid_t dset_id   = H5I_INVALID_HID;
    hid_t space_id  = H5I_INVALID_HID;

    file_id = H5Fcreate("file.h5", H5F_ACC_TRUNC,
                        H5P_DEFAULT, H5P_DEFAULT);

    space_id = H5Screate_simple(2, dims, NULL);

    dset_id = H5Dcreate2(file_id, "dset", H5T_IEEE_F16LE,
                         space_id, H5P_DEFAULT,
                         H5P_DEFAULT, H5P_DEFAULT);

    ...

    H5Sclose(space_id);
    H5Dclose(dset_id);
    H5Fclose(file_id);

    return 0;
}
\end{minted}

\newpage

\textbf{Write to a binary16 little-endian 16-bit floating-point dataset using native \texttt{\_Float16} type}

\begin{minted}[custom lexer]{hdf5-c-lexer.py:HDF5CLexer}
#include "hdf5.h"

int
main(int argc, char **argv)
{
    _Float16 data[100];
    hsize_t dims[2] = { 10, 10 };
    hid_t file_id   = H5I_INVALID_HID;
    hid_t dset_id   = H5I_INVALID_HID;
    hid_t space_id  = H5I_INVALID_HID;

    file_id = H5Fcreate("file.h5", H5F_ACC_TRUNC,
                        H5P_DEFAULT, H5P_DEFAULT);

    space_id = H5Screate_simple(2, dims, NULL);

    dset_id = H5Dcreate2(file_id, "dset", H5T_IEEE_F16LE,
                         space_id, H5P_DEFAULT,
                         H5P_DEFAULT, H5P_DEFAULT);

    for (size_t i = 0; i < 100; i++)
        data[i] = 1.1f16;

    H5Dwrite(dset_id, H5T_NATIVE_FLOAT16, H5S_ALL,
             H5S_ALL, H5P_DEFAULT, data);

    H5Sclose(space_id);
    H5Dclose(dset_id);
    H5Fclose(file_id);

    return 0;
}
\end{minted}

\newpage

\textbf{Write to and read from a binary16 little-endian 16-bit floating-point dataset using native \texttt{\_Float16} type}

\begin{minted}[custom lexer]{hdf5-c-lexer.py:HDF5CLexer}
#include "hdf5.h"

int
main(int argc, char **argv)
{
    _Float16 data[100];
    _Float16 read_buf[100];
    hsize_t dims[2] = { 10, 10 };
    hid_t file_id   = H5I_INVALID_HID;
    hid_t dset_id   = H5I_INVALID_HID;
    hid_t space_id  = H5I_INVALID_HID;

    file_id = H5Fcreate("file.h5", H5F_ACC_TRUNC,
                        H5P_DEFAULT, H5P_DEFAULT);

    space_id = H5Screate_simple(2, dims, NULL);

    dset_id = H5Dcreate2(file_id, "dset", H5T_IEEE_F16LE,
                         space_id, H5P_DEFAULT,
                         H5P_DEFAULT, H5P_DEFAULT);

    for (size_t i = 0; i < 100; i++)
        data[i] = 1.1f16;

    H5Dwrite(dset_id, H5T_NATIVE_FLOAT16, H5S_ALL,
             H5S_ALL, H5P_DEFAULT, data);

    H5Dread(dset_id, H5T_NATIVE_FLOAT16, H5S_ALL,
            H5S_ALL, H5P_DEFAULT, read_buf);

    for (size_t i = 0; i < 100; i++)
        printf("%f\n", (double)read_buf[i]);

    H5Sclose(space_id);
    H5Dclose(dset_id);
    H5Fclose(file_id);

    return 0;
}
\end{minted}

\newpage

\textbf{Write to and read from a little-endian 128-bit (2 64-bit doubles) complex number dataset using native \texttt{double \_Complex} type}

\begin{minted}[custom lexer]{hdf5-c-lexer.py:HDF5CLexer}
#include <complex.h>

#include "hdf5.h"

int
main(int argc, char **argv)
{
    double _Complex data[100];
    double _Complex read_buf[100];
    hsize_t dims[2] = { 10, 10 };
    hid_t file_id   = H5I_INVALID_HID;
    hid_t dset_id   = H5I_INVALID_HID;
    hid_t space_id  = H5I_INVALID_HID;

    file_id = H5Fcreate("file.h5", H5F_ACC_TRUNC,
                        H5P_DEFAULT, H5P_DEFAULT);

    space_id = H5Screate_simple(2, dims, NULL);

    dset_id = H5Dcreate2(file_id, "dset", H5T_STD_DOUBLE_COMPLEX_LE,
                         space_id, H5P_DEFAULT,
                         H5P_DEFAULT, H5P_DEFAULT);

    for (size_t i = 0; i < 100; i++)
        data[i] = 1.0 + 2.0*I;

    H5Dwrite(dset_id, H5T_NATIVE_DOUBLE_COMPLEX,
             H5S_ALL, H5S_ALL, H5P_DEFAULT, data);

    H5Dread(dset_id, H5T_NATIVE_DOUBLE_COMPLEX,
            H5S_ALL, H5S_ALL, H5P_DEFAULT, read_buf);

    for (size_t i = 0; i < 100; i++)
        printf("%f%+fi\n", creal(read_buf[i]), cimag(read_buf[i]));

    H5Sclose(space_id);
    H5Dclose(dset_id);
    H5Fclose(file_id);

    return 0;
}
\end{minted}

\newpage

\textbf{Write to a little-endian 128-bit (2 64-bit doubles) complex number dataset when platform doesn't support \texttt{double \_Complex} type}

\begin{minted}[custom lexer]{hdf5-c-lexer.py:HDF5CLexer}
#include "hdf5.h"

int
main(int argc, char **argv)
{
    /* Note that data has twice the number of elements as the dataset */
    double data[200];
    hsize_t dims[2] = { 10, 10 };
    hid_t file_id   = H5I_INVALID_HID;
    hid_t dset_id   = H5I_INVALID_HID;
    hid_t space_id  = H5I_INVALID_HID;

    file_id = H5Fcreate("file.h5", H5F_ACC_TRUNC,
                        H5P_DEFAULT, H5P_DEFAULT);

    space_id = H5Screate_simple(2, dims, NULL);

    dset_id = H5Dcreate2(file_id, "dset", H5T_STD_DOUBLE_COMPLEX_LE,
                         space_id, H5P_DEFAULT,
                         H5P_DEFAULT, H5P_DEFAULT);

    for (size_t i = 0; i < 200; ) {
        data[i]     = 1.0; /* real part */
        data[i + 1] = 2.0; /* imaginary part */
        i += 2;
    }

    H5Dwrite(dset_id, H5T_STD_DOUBLE_COMPLEX_LE,
             H5S_ALL, H5S_ALL, H5P_DEFAULT, data);

    H5Sclose(space_id);
    H5Dclose(dset_id);
    H5Fclose(file_id);

    return 0;
}
\end{minted}

\end{document}