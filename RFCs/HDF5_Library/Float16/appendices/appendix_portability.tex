\documentclass[../HDF5_RFC.tex]{subfiles}
 
\begin{document}

% Start each appendix on a new page
\newpage

\section{Appendix: Portability}
\label{apdx:portability}

The following items must be taken into account for supporting these new datatypes across different
platforms and compilers.

\subsection{\texttt{\_Float16}}

\subsubsection{GCC}

This type is supported by GCC on:

\begin{itemize}
    \item x86 systems where SSE2 support is available
    \item ARM systems when the \texttt{-mfp16-format=ieee} flag is used to specify that the IEEE
          half-precision floating-point format should be used
    \item AArch64 systems by default
\end{itemize}

\subsubsection{Clang}

Based on \href{https://clang.llvm.org/docs/LanguageExtensions.html#half-precision-floating-point}{Clang Language Extensions},
this type is supported by Clang on:

\begin{itemize}
    \item x86 systems (natively if AVX512-FP16 is available; emulated otherwise)
    \item 32-bit ARM systems (emulated on some architectures; natively on some architectures)
    \item AArch64 systems (natively on ARMv8.2a and later)
    \item AMDGPU (natively)
    \item RISC-V (natively if Zfh or Zhinx support is available)
\end{itemize}

where "natively" means the processor can directly perform arithmetic on the type using
dedicated instructions and "emulated" means the compiler will promote to \texttt{float}
before performing arithmetic.

\subsubsection{MSVC}

As of the time of writing, Visual Studio 2022 does not appear to have support for any half-precision floating-point types.

\subsubsection{NVHPC}

NVHPC appears to support this type when compiling code used for CUDA compute (e.g., with \texttt{nvcc}), but
does not appear to support this type when compiling regular C code with \texttt{nvc}.

\subsubsection{OneAPI}

Based on \href{https://www.intel.com/content/www/us/en/docs/dpcpp-cpp-compiler/developer-guide-reference/2024-0/math-function-list.html}{Intel oneAPI Math Function List}, this type appears to be supported as of oneAPI version 2021.4 or higher with a "next-generation Intel® Xeon® Scalable processor".
The \href{https://www.intel.com/content/www/us/en/developer/tools/oneapi/onednn.html#gs.3ikawv}{Intel oneAPI Deep Neural Network Library} also appears
to have support for a 16-bit floating-point type through the \href{https://www.intel.com/content/www/us/en/docs/onednn/developer-guide-reference/2024-0/data-types-001.html}{f16} type specifier, but it's not clear whether that type is more or less appropriate as an interchange format.

\subsection{\texttt{\_\_bf16}}

\subsubsection{GCC}

This type is supported as of GCC 13 on:

\begin{itemize}
    \item x86 systems where SSE2 support is available
\end{itemize}

\subsubsection{Clang}

Based on \href{https://clang.llvm.org/docs/LanguageExtensions.html#half-precision-floating-point}{Clang Language Extensions},
this type is supported by Clang on:

\begin{itemize}
    \item x86 systems where SSE2 support is available
    \item 32-bit ARM systems
    \item AArch64 systems
    \item RISC-V systems
\end{itemize}

Note that support for the \texttt{\_\_bf16} type is currently always emulated and not
supported natively on any of the platforms above.

\subsubsection{MSVC}

As of the time of writing, Visual Studio 2022 does not appear to have support for any half-precision floating-point types.

\subsubsection{NVHPC}

This type appears to be supported by NVHPC as of version 23.9.

\subsubsection{OneAPI}

The \href{https://www.intel.com/content/www/us/en/developer/tools/oneapi/onednn.html#gs.3ikawv}{Intel oneAPI Deep Neural Network Library}
(oneDNN) appears to have support for this type through the \texttt{bf16} type specifier. A whitepaper on the topic can be found at
\href{https://www.intel.com/content/dam/develop/external/us/en/documents/bf16-hardware-numerics-definition-white-paper.pdf}{here}.

\subsection{\texttt{\_Bool}}

This type is supported on any C99-compliant compiler, which includes all the compilers mentioned for other types here.

\subsection{\texttt{float \_Complex}, \texttt{double \_Complex}, \texttt{long double \_Complex}}

These types will generally be supported on any platform and compiler combination that
implements the ISO C99 standard. However, it should be noted that support for complex
numbers became optional in the C11 standard, meaning that C99 compliance is not enough
of an indicator of support for these types. As of the C11 standard, there is a feature
test macro, \texttt{\_\_STDC\_NO\_COMPLEX\_\_}, to determine this support, but as it only
exists in the C11 standard and HDF5 only requires C99 compliance, additional configure
time checks will likely be needed to check for complex number support. As of the C99
standard, there is also the feature test macro, \texttt{\_\_STDC\_IEC\_559\_COMPLEX\_\_},
which determines whether the available complex number support conforms to IEC 60559.
Notably, this involves supporting some additional macros and types for imaginary numbers.
While in theory this macro could be used to check for complex number support, it is only
a recommendation in the C99 standard that a compiler which defines this macro should support
imaginary numbers. As of the C11 standard, it became mandatory for a compiler to support
imaginary numbers if it defines this macro, but as of the C23 standard this macro has
changed to \texttt{\_\_STDC\_IEC\_60559\_COMPLEX\_\_}. Therefore, a configure time check
for complex number support should likely rely on trying to compile a program that uses
the complex number types rather than relying on these macros (though \texttt{\_\_STDC\_NO\_COMPLEX\_\_}
can be checked if C11 support or later is available).

\subsubsection{GCC}

These types are fully supported, as long as the value of \texttt{\_\_STDC\_NO\_COMPLEX\_\_},
if available, is not $1$.

\subsubsection{Clang}

These types are fully supported, as long as the value of \texttt{\_\_STDC\_NO\_COMPLEX\_\_},
if available, is not $1$.

\subsubsection{MSVC}

\href{https://learn.microsoft.com/en-us/cpp/c-runtime-library/complex-math-support?view=msvc-170}{Support for complex numbers is available},
but MSVC does not expose the \texttt{float \_Complex}, \texttt{double \_Complex}, \texttt{long double \_Complex} types and instead uses
the \texttt{\_Fcomplex}, \texttt{\_Dcomplex}, \texttt{\_Lcomplex} types, respectively.

\subsubsection{NVHPC}

These types are fully supported, as long as the value of \texttt{\_\_STDC\_NO\_COMPLEX\_\_},
if available, is not $1$.

\subsubsection{OneAPI}

As oneAPI is C99 compliant, these types should be fully supported, as long as the value of
\texttt{\_\_STDC\_NO\_COMPLEX\_\_}, if available, is not $1$.

\end{document}