\documentclass[../HDF5_RFC.tex]{subfiles}
 
\begin{document}

\section{Approach}
\label{approach}

\subsection{Configure time checks}

For each new datatype added, configure-time checks will be added to both CMake and the Autotools to
detect the size of the native form of each datatype. New macros in the form of \texttt{H5\_SIZEOF\_type}
will be added to the files that generate the \texttt{H5pubconf.h} header file. If a compiler doesn't
support a particular type, the macro for that type will have a value of $0$.

\subsection{New datatype IDs and macros}

For each new datatype added, variables will be added to \texttt{H5T.c} for the IDs of library-internal
datatypes which will represent the platform-native form of each datatype, as well as the "standard" file
format storage form of each datatype. Then, the following new macros will be added for the new datatypes:

\subsubsection{\_Float16}

\texttt{H5T\_NATIVE\_FLOAT16}

Maps to the ID of the library-internal datatype representing the platform's native representation
of an IEEE binary16-format 16-bit floating point number.

\textbf{NOTE: If the platform doesn't support the \texttt{\_Float16} type, this macro will map to
\texttt{H5I\_INVALID\_HID} and will cause an "invalid datatype" error if used for any HDF5 operation
which is passed a datatype identifier. In this case, one should use the \texttt{H5T\_IEEE\_F16LE} or
\texttt{H5T\_IEEE\_F16BE} macro when writing from or reading into buffers of binary16-format data.}

\texttt{H5T\_IEEE\_F16LE}

Maps to the ID of the library-internal datatype representing a binary16-format 16-bit
little-endian IEEE floating-point number.

\texttt{H5T\_IEEE\_F16BE}

Maps to the ID of the library-internal datatype representing a binary16-format 16-bit
big-endian IEEE floating-point number.

\subsubsection{\_\_bf16}

\texttt{H5T\_NATIVE\_BFLOAT16}

Maps to the ID of the library-internal datatype representing the platform's native representation
of a bfloat16-format 16-bit floating point number.

\textbf{NOTE: If the platform doesn't support the \texttt{\_\_bf16} type, this macro will map to
\texttt{H5I\_INVALID\_HID} and will cause an "invalid datatype" error if used for any HDF5 operation
which is passed a datatype identifier. In this case, one should use the \texttt{H5T\_BRAIN\_F16LE} or
\texttt{H5T\_BRAIN\_F16BE} macro when writing from or reading into buffers of bfloat16-format data.}

\texttt{H5T\_BRAIN\_F16LE}

Maps to the ID of the library-internal datatype representing a bfloat16-format 16-bit
little-endian floating-point number.

\texttt{H5T\_BRAIN\_F16BE}

Maps to the ID of the library-internal datatype representing a bfloat16-format 16-bit
big-endian floating-point number.

\subsubsection{\texttt{\_Bool}}

\texttt{H5T\_NATIVE\_BOOL}

Maps to the ID of the library-internal datatype representing the platform's native representation
of a \texttt{\_Bool} boolean type.

\texttt{H5T\_STD\_BOOL}

Maps to the ID of the library-internal datatype representing a 1-bit boolean value.

\subsubsection{\texttt{float \_Complex}, \texttt{double \_Complex}, \texttt{long double \_Complex}}

\texttt{H5T\_NATIVE\_FLOAT\_COMPLEX}

Maps to the ID of the library-internal datatype representing the platform's native representation
of a complex number consisting of \texttt{float} values.

\textbf{NOTE: If the platform doesn't support the \texttt{float \_Complex} type, this macro will map to
\texttt{H5I\_INVALID\_HID} and will cause an "invalid datatype" error if used for any HDF5 operation
which is passed a datatype identifier. In this case, one should use the \texttt{H5T\_STD\_FLOAT\_COMPLEX\_LE}
or \texttt{H5T\_STD\_FLOAT\_COMPLEX\_BE} macro when writing from or reading into buffers of complex numbers
consisting of 32-bit-precision floating-point values.}

\texttt{H5T\_STD\_FLOAT\_COMPLEX\_LE}

Maps to the ID of the library-internal datatype representing a complex number consisting of 2
32-bit-precision little-endian floating-point values.

\texttt{H5T\_STD\_FLOAT\_COMPLEX\_BE}

Maps to the ID of the library-internal datatype representing a complex number consisting of 2
32-bit-precision big-endian floating-point values.

\texttt{H5T\_NATIVE\_DOUBLE\_COMPLEX}

Maps to the ID of the library-internal datatype representing the platform's native representation
of a complex number consisting of \texttt{double} values.

\textbf{NOTE: If the platform doesn't support the \texttt{double \_Complex} type, this macro will map to
\texttt{H5I\_INVALID\_HID} and will cause an "invalid datatype" error if used for any HDF5 operation
which is passed a datatype identifier. In this case, one should use the \texttt{H5T\_STD\_DOUBLE\_COMPLEX\_LE}
or \texttt{H5T\_STD\_DOUBLE\_COMPLEX\_BE} macro when writing from or reading into buffers of complex numbers
consisting of 64-bit-precision floating-point values.}

\texttt{H5T\_STD\_DOUBLE\_COMPLEX\_LE}

Maps to the ID of the library-internal datatype representing a complex number consisting of 2
64-bit-precision little-endian floating-point values.

\texttt{H5T\_STD\_DOUBLE\_COMPLEX\_BE}

Maps to the ID of the library-internal datatype representing a complex number consisting of 2
64-bit-precision big-endian floating-point values.

\texttt{H5T\_NATIVE\_LDOUBLE\_COMPLEX}

Maps to the ID of the library-internal datatype representing the platform's native representation
of a complex number consisting of \texttt{long double} values.

\textbf{NOTE: If the platform doesn't support the \texttt{long double \_Complex} type, this macro will
map to \texttt{H5I\_INVALID\_HID} and will cause an "invalid datatype" error if used for any HDF5
operation which is passed a datatype identifier. In this case, one should use the \texttt{H5T\_STD\_LDOUBLE\_COMPLEX\_LE}
or \texttt{H5T\_STD\_LDOUBLE\_COMPLEX\_BE} macro when writing from or reading into buffers of complex numbers
consisting of 128-bit-precision floating-point values.}

\texttt{H5T\_STD\_LDOUBLE\_COMPLEX\_LE}

Maps to the ID of the library-internal datatype representing a complex number consisting of 2
128-bit-precision little-endian floating-point values.

\texttt{H5T\_STD\_LDOUBLE\_COMPLEX\_BE}

Maps to the ID of the library-internal datatype representing a complex number consisting of 2
128-bit-precision big-endian floating-point values.

\subsection{On-disk format for new datatypes}

\subsubsection{\texttt{\_Float16}}

Data of this type will be stored as 16-bit values on disk, according to the endian-ness of the specified
datatype (e.g. \texttt{H5T\_IEEE\_F16LE} vs. \texttt{H5T\_IEEE\_F16BE}), in a similar format to how other
floating-point values are stored in HDF5.

\subsubsection{\texttt{\_\_bf16}}

Data of this type will be stored as 16-bit values on disk, according to the endian-ness of the specified
datatype (e.g. \texttt{H5T\_BRAIN\_F16LE} vs. \texttt{H5T\_BRAIN\_F16BE}), in a similar format to how other
floating-point values are stored in HDF5.

\subsubsection{\texttt{\_Bool}}

Data of this type will be stored as a 1-bit value on disk.

\subsubsection{\texttt{float \_Complex}, \texttt{double \_Complex}, \texttt{long double \_Complex}}

Data of these types will be stored as atomic 64-bit, 128-bit, or 256-bit values on disk, respectively,
according to the endian-ness of the specified datatype. The benefit to choosing this format is that
these values should be able to be converted directly from disk format to a platform's in-memory
representation with zero or near-zero overhead during conversion (depending on the endian-ness of the disk
format and the in-memory representation). However, this means that there will likely be some overhead during
conversion from disk format to in-memory representation for applications that use their own convention, such
as a compound datatype with two members, for representing complex numbers.

\textbf{NOTE: These on-disk formats have been chosen as they map to common sizes (multiplied by 2) of \texttt{float},
\texttt{double} and \texttt{long double} across platforms. Care should be taken when interacting with these
datatypes on a platform where the size of the C types don't match these sizes.}

\subsection{Conversions}

For each new datatype added, hard and soft conversion routines will be added to \texttt{H5Tconv.c} to
convert between the new type and each existing native integer and floating-point type. In particular,
the conversions for complex number types will follow the rules laid out in
\href{https://github.com/HDFGroup/hdf5doc/blob/master/RFCs/HDF5_Library/New_Datatypes/new_datatypes.pdf}{RFC: New Datatypes}.
Additionally, some conversions between complex number types and existing user-defined representations
(e.g., a compound type with two members, an array type with two elements, etc.) will be added to help
maintain compatibility with those representations.

\subsection{\texttt{H5trace}}

For each new datatype added, a printout for the new datatype will be added to the \texttt{H5\_trace\_args}
function in \texttt{H5trace.c}.

\subsection{Testing}

For each new datatype added, tests will be added to the library to exercise both using the new datatype
to create objects, as well as converting between the new datatype and existing HDF5 datatypes
(see \texttt{test/dt\_arith.c}).

\end{document}