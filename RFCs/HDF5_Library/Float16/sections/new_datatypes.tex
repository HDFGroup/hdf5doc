\documentclass[../HDF5_RFC.tex]{subfiles}
 
\begin{document}

\section{New datatypes}
\label{new_datatypes}

Support for the following C datatypes is being considered for addition to the HDF5 library:
\texttt{\_Float16}, \texttt{\_\_bf16}, \texttt{\_Bool}, \texttt{float \_Complex}, \texttt{double \_Complex}, \texttt{long double \_Complex}.

\subsection{\texttt{\_Float16}}

This is a half-precision floating-point binary interchange format type specified in \textbf{ISO/IEC TS 18661-3:2015}.
It represents a \href{https://en.wikipedia.org/wiki/Half-precision_floating-point_format}{binary16-format} floating-point
type consisting of:

\begin{itemize}
    \item 16 bits of storage
    \item 1 bit for the sign
    \item 5 bits for the exponent
    \item 10 bits for the fraction
\end{itemize}

An alternative to this type is the \texttt{\_\_fp16} type, but arithmetic operations on this type promote to \texttt{float}
or higher, while arithmetic can often be performed directly on values of the \texttt{\_Float16} type. It is also generally
recommended that portable code use the \texttt{\_Float16} type, as it is standardized by the C standards committee. Due to
this, only support for the \texttt{\_Float16} type is being considered here; support for the \texttt{\_\_fp16} type may be
considered in the future.

\subsection{\texttt{\_\_bf16}}

This is a half-precision floating-point type developed by \href{https://en.wikipedia.org/wiki/Google_Brain}{Google Brain}
which is useful for machine learning applications and other workflows where the reduced storage requirements and potentially
faster calculation speeds of the format are important. It represents a \href{https://en.wikipedia.org/wiki/Bfloat16_floating-point_format}{bfloat16-format}
floating-point type consisting of:

\begin{itemize}
    \item 16 bits of storage
    \item 1 bit for the sign
    \item 8 bits for the exponent
    \item 7 bits for the fraction
\end{itemize}

This format has not been formally standardized by the C standards committee and support varies across platforms, so it
may be tricky to support this format.

\subsection{\texttt{\_Bool}}

This is the boolean datatype introduced in the C99 standard. The main reason for including native support for this type
in HDF5 is to be able to reduce its storage requirements by not having to rely on integer types for storing boolean
values.

\subsection{\texttt{float \_Complex}, \texttt{double \_Complex}, \texttt{long double \_Complex}}

These are types representing complex numbers consisting of a real and imaginary part, each of which is a floating-point
value of the specified type (\texttt{float}, \texttt{double} or \texttt{long double}). These types have been discussed
more extensively in \href{https://github.com/HDFGroup/hdf5doc/blob/master/RFCs/HDF5_Library/New_Datatypes/new_datatypes.pdf}{RFC: New Datatypes}
and are simply brought up once again for discussion around inclusion into HDF5.

\subsection{Other lesser-precision floating-point datatypes}

Consideration has also been given towards supporting lesser-precision floating-point datatypes, such as 8-bit floating-point
datatypes. However, current compiler support for this type seems to be minimal or nonexistent across a variety of platform/compiler
combinations. Therefore, support for these types would not be implemented at this time, but may be considered in the future
if they are found to be useful.

\end{document}