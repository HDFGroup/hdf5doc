\documentclass[../HDF5_RFC.tex]{subfiles}
 
\begin{document}

\section{Introduction}
\label{intro}

HDF5 has a rich set of predefined datatypes to cover some of the most common use cases when
storing data in HDF5 files. However, there are several types of data in wide use today that
HDF5 does not handle in a convenient or efficient manner, causing users to have to workaround
this lack of support in ways that are often cumbersome.

One of the goals that this RFC intends to accomplish is enabling convenient and efficient
storage and retrieval of half-precision floating-point data in HDF5 files. An HDF5 user can
store their data in a dataset using one of the existing (at least 32-bit) floating-point
types and later convert the data to half-precision floating-point in memory for calculations,
but as the reduced storage requirements for 16-bit floating-point values is one of the main
benefits of the datatype, this isn't an ideal scenario. While one can already create an HDF5
datatype that will allow storing of floating-point data as 16-bit values on disk, the process
of doing so with HDF5's \texttt{H5T} API routines may not be obvious to users, meaning that
applications could potentially benefit from having a convenient predefined datatype to use.
Further, datatype conversions on datatypes like this which HDF5 doesn't have native support
for can have very poor performance, as demonstrated in issue \href{https://github.com/HDFGroup/hdf5/issues/2154}{\#2154}
in HDF5's GitHub repository. This is due to HDF5 performing conversions on these datatypes
in software rather than allowing the compiler to potentially perform these conversions in
hardware through dedicated instructions. Once again, an HDF5 user could work around this issue
by defining their own datatype conversion routine, but it would be better and more convenient
for HDF5 to be able to perform fast conversions by default where possible.

Another problematic topic in the world of HDF5 data involves the storage of complex numbers.
Complex numbers are used throughout the scientific and other communities, but HDF5 does not
currently have native support for them. This has led HDF5 users to invent their own conventions
for storing this type of data, which can be inconvenient and can result in HDF5 files that have
poor interoperability. A \href{https://github.com/HDFGroup/hdf5/discussions/3339}{feature request}
for native complex number support in HDF5 was made fairly recently and contains a good discussion
on the issues surrounding this problem.

To address these issues and work toward more portable HDF5 files, this RFC proposes adding native
support in HDF5 for several datatypes that have been requested in the past, as well as datatypes
that will be useful in future workflows which use HDF5.

\end{document}